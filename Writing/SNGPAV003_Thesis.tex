%%%%%%%%%%%%%%%%%%%%%%%%%%%%%%%%%%%%%%%%%
% Masters/Doctoral Thesis 
% LaTeX Template
% Version 2.5 (27/8/17)
%
% This template was downloaded from:
% http://www.LaTeXTemplates.com
%
% Version 2.x major modifications by:
% Vel (vel@latextemplates.com)
%
% This template is based on a template by:
% Steve Gunn (http://users.ecs.soton.ac.uk/srg/softwaretools/document/templates/)
% Sunil Patel (http://www.sunilpatel.co.uk/thesis-template/)
%
% Template license:
% CC BY-NC-SA 3.0 (http://creativecommons.org/licenses/by-nc-sa/3.0/)
%
%%%%%%%%%%%%%%%%%%%%%%%%%%%%%%%%%%%%%%%%%

%----------------------------------------------------------------------------------------
%	PACKAGES AND OTHER DOCUMENT CONFIGURATIONS
%----------------------------------------------------------------------------------------

\documentclass[
12pt, % The default document font size, options: 10pt, 11pt, 12pt
english, % ngerman for German
singlespacing, % Single line spacing, alternatives: onehalfspacing or doublespacing
%draft, % Uncomment to enable draft mode (no pictures, no links, overfull hboxes indicated)
%liststotoc, % Uncomment to add the list of figures/tables/etc to the table of contents
%toctotoc, % Uncomment to add the main table of contents to the table of contents
parskip, % Uncomment to add space between paragraphs
%nohyperref, % Uncomment to not load the hyperref package
headsepline, % Uncomment to get a line under the header
%chapterinoneline, % Uncomment to place the chapter title next to the number on one line
% consistentlayout, % Uncomment to change the layout of the declaration, abstract and acknowledgements pages to match the default layout
openany,
]{MastersDoctoralThesis} % The class file specifying the document structure

\usepackage[utf8]{inputenc} % Required for inputting international characters
\usepackage[T1]{fontenc} % Output font encoding for international characters

\usepackage{mathpazo} % Use the Palatino font by default
\usepackage{tabularx}
\usepackage[demo]{graphicx}
\usepackage{subcaption}
\usepackage{changepage}
\usepackage{amssymb}
\usepackage{amsmath}
\usepackage{multirow}
\usepackage[boxed]{algorithm}
\usepackage{enumerate}
\usepackage{algpseudocode}
\usepackage{float}
\usepackage{bbm}
\usepackage{multirow}

\usepackage{biblatex}
\addbibresource{/Users/pavansingh/Google Drive (UCT)/STA Honours/Project/Thesis/Write-Up/Draft/Thesis.bib}

\renewcommand{\familydefault}{\rmdefault}

%\renewcommand{\familydefault}{\sfdefault}

\usepackage{natbib}

\renewcommand*{\bibfont}{\footnotesize}

\usepackage[autostyle = true]{csquotes} % Required to generate language-dependent quotes in the bibliography

\usepackage[bookmarksopen=true,
bookmarksopenlevel=0,
hypertexnames=false,
colorlinks=true,% Set to false to disable coloring links
citecolor=blue,% The color of citations
linkcolor=mdtRed,% The color of references to document elements (sections, figures, etc)
urlcolor=mdtRed,% The color of hyperlinks (URLs)
breaklinks=true]{hyperref}

\usepackage{cleveref}

\newtheorem{theorem}{Theorem}
\numberwithin{theorem}{section}
\newtheorem{remark}{Remark}
\numberwithin{remark}{section}
\newtheorem{assumption}{Assumption}
\numberwithin{assumption}{section}

\renewcommand{\vec}[1]{\mathbf{#1}}
\newcommand{\iu}{\mathrm{i}\mkern1mu}
\DeclareMathAlphabet\mathbfcal{OMS}{cmsy}{b}{n}

\newcommand\hmmax{0}
\newcommand\bmmax{0}

\usepackage{physics}
\usepackage{amsmath}
\usepackage{tikz}
\usepackage{mathdots}
\usepackage{yhmath}
\usepackage{cancel}
\usepackage{color}
\usepackage{siunitx}
\usepackage{array}
\usepackage{multirow}
\usepackage{amssymb}
% \usepackage{gensymb}
\usepackage{tabularx}
\usepackage{booktabs}
\usepackage{mathptmx}
% \usepackage{bbm}
\usetikzlibrary{fadings}
\usetikzlibrary{patterns}
\usetikzlibrary{shadows.blur}
\usetikzlibrary{shapes}
% \usepackage{subfig} %---------------- for subfigures

\usepackage{caption3} % load caption package kernel first
\DeclareCaptionOption{parskip}[]{} % disable "parskip" caption option
% \usepackage[small]{caption}

\usepackage[T1]{fontenc}
\usepackage[utf8]{inputenc}
\pagestyle{plain}
\setcounter{secnumdepth}{3}
\setcounter{tocdepth}{3}
\setlength{\parskip}{\medskipamount}
\setlength{\parindent}{0pt}
\makeatletter
\usepackage{caption}%----------------------- added newly
%\captionsetup{labelfont={sc,bf}}%----------------------- added newly
\captionsetup[figure]{labelfont={bf,footnotesize,rm},textfont={rm,footnotesize}}%----------------------- added newly
\captionsetup[subfloat]{labelfont={bf,footnotesize},textfont={rm,footnotesize}}%----------------------- added newly
\captionsetup[algorithm]{labelfont={bf,footnotesize},textfont={rm,footnotesize}}%----------------------- added newly
\captionsetup[table]{labelfont={bf,footnotesize},textfont={rm,footnotesize}}%----------------------- added newly

%----------------------------------------------------------------------------------------
%	MARGIN SETTINGS
%----------------------------------------------------------------------------------------

\geometry{
	paper=a4paper, % Change to letterpaper for US letter
	inner=1.0cm, % Inner margin
	outer=2.0cm, % Outer margin
	bindingoffset=.5cm, % Binding offset
	top=1.0cm, % Top margin
	bottom=1.0cm, % Bottom margin
	%showframe, % Uncomment to show how the type block is set on the page
}

%----------------------------------------------------------------------------------------
%	THESIS INFORMATION
%----------------------------------------------------------------------------------------
% Modelling high-frequency correlation dynamics: Is the Epps effect a bias?
\thesistitle{Comparison of Non-Linear Machine Learning Methods on Non-Linear Time Series Data} % Your thesis title, this is used in the title and abstract, print it elsewhere with \ttitle
\supervisor{Dr. Birgit \textsc{Ernie} \\
		Dr. Etienne \textsc{Pienaar}} % Your supervisor's name, this is used in the title page, print it elsewhere with \supname
\examiner{} % Your examiner's name, this is not currently used anywhere in the template, print it elsewhere with \examname
\course{Bachelor of Business Science Analytics} % Your degree name, this is used in the title page and abstract, print it elsewhere with \degreename
\author{Pavan \textsc{Singh}} % Your name, this is used in the title page and abstract, print it elsewhere with \authorname
\addresses{} % Your address, this is not currently used anywhere in the template, print it elsewhere with \addressname

\subject{Statistical Science Honours} % Your subject area, this is not currently used anywhere in the template, print it elsewhere with \subjectname
\keywords{} % Keywords for your thesis, this is not currently used anywhere in the template, print it elsewhere with \keywordnames
\university{\href{http://www.uct.ac.za/}{University of Cape Town}} % Your university's name and URL, this is used in the title page and abstract, print it elsewhere with \univname
\department{\href{http://www.stats.uct.ac.za/}{Department of Statistical Sciences}} % Your department's name and URL, this is used in the title page and abstract, print it elsewhere with \deptname
% \group{\href{http://researchgroup.university.com}{Research Group Name}} % Your research group's name and URL, this is used in the title page, print it elsewhere with \groupname
\faculty{\href{http://faculty.university.com}{Science}} % Your faculty's name and URL, this is used in the title page and abstract, print it elsewhere with \facname

\AtBeginDocument{
\hypersetup{pdftitle=\ttitle} % Set the PDF's title to your title
\hypersetup{pdfauthor=\authorname} % Set the PDF's author to your name
\hypersetup{pdfkeywords=\keywordnames} % Set the PDF's keywords to your keywords
}

\begin{document}

\frontmatter % Use roman page numbering style (i, ii, iii, iv...) for the pre-content pages

\pagestyle{plain} % Default to the plain heading style until the thesis style is called for the body content

%----------------------------------------------------------------------------------------
%	TITLE PAGE
%----------------------------------------------------------------------------------------

\begin{titlepage}
\begin{center}

\includegraphics*[width=\linewidth]{Figures/UCTLogoLong.jpg} 

\vspace*{.06\textheight}
{\scshape\LARGE \univname\par}\vspace{1.0cm} % University name
\textsc{\Large Honour's Thesis}\\[0.5cm] % Thesis type

\HRule \\[0.4cm] % Horizontal line
{\large \bfseries \ttitle\par}\vspace{0.4cm} % Thesis title
\HRule \\[1.5cm] % Horizontal line
 
\begin{minipage}[t]{0.4\textwidth}
\begin{flushleft} \large
{Author:}\\
\authorname
\end{flushleft}
\end{minipage}
\begin{minipage}[t]{0.4\textwidth}
\begin{flushright} \large
{Supervisor:} \\
\supname
\end{flushright}
\end{minipage}\\[1cm]
 
\vfill

\large {A thesis presented for the degree of \\ \degreename \ }\\[0.3cm] % University requirement text
{from the}\\[0.4cm]
\deptname\\[1cm] % Research group name and department name
\includegraphics*[width=0.25\linewidth]{Figures/statslogo.png} \\[1cm]
{\large \today}\\[4cm] % Date
%\includegraphics{Logo} % University/department logo - uncomment to place it
 
\vfill
\end{center}
\end{titlepage}

%----------------------------------------------------------------------------------------
%	ACKNOWLEDGEMENTS
%----------------------------------------------------------------------------------------
\begin{acknowledgements}
\addchaptertocentry{\acknowledgementname} 
\vspace{1.5cm}

I would like to extend my deepest gratitude to my parents and sister who have provided unwavering support throughout my academic studies.

\vspace{0.3cm}

I would like to thank Dr. Birgit Ernie for her timely feedback and assistance throughout this project, particularly for her constant guidance, patience and support this year. I also would like to thank her for accepting and offering to supervise my project. I would also like to extend my thanks Dr. Etienne Pienaar for offering to co-supervise my project and also give me the opportunity to conduct my own research into a topic that interests me.   

\vspace{0.3cm}

I would additionally like to acknowledge the assistance and support from my friends and fellow statistics students, who have provided persistent support and advice during this year.

\end{acknowledgements}


%----------------------------------------------------------------------------------------
%	DECLARATION PAGE
%----------------------------------------------------------------------------------------

\begin{declaration}
\addchaptertocentry{\authorshipname} 
\vspace{1.5cm}

\noindent I, \authorname, declare that this thesis titled, \enquote{\ttitle} and the work presented in it are my own. I confirm that:

\begin{itemize} 
\item This work was done wholly while in candidature for a research degree at this University.
\item The contents of this thesis has not been previously submitted for a degree or any other qualification at this University or any other institution.
\item Where I have consulted the published work of others, this is always clearly attributed.
\item Where I have quoted from the work of others, the source is always given. With the exception of such quotations, this thesis is entirely my own work.
\item I have acknowledged all main sources of help.
\end{itemize}

\vspace{1cm}
\noindent Signed:\\
\rule[0.5em]{25em}{0.5pt} % This prints a line for the signature
 
\noindent Date:\\
\rule[0.5em]{25em}{0.5pt} % This prints a line to write the date
\end{declaration}

%----------------------------------------------------------------------------------------
%	ABSTRACT PAGE
%----------------------------------------------------------------------------------------

\begin{abstract}
\addchaptertocentry{\abstractname} % Add the abstract to the table of contents

Forecasting is an indispensable tool in the modern era, which has applications in various fields, from business to health sciences. Inherent volatility and non-linearity in data can make accurate forecasting a challenging task. These properties intrinsically characterise financial data, in particular stock market data. Machine learning methods like artificial neural networks have become increasingly popular for tackling a range of problems in the financial domain, due to their ability to discern complex underlying patterns and capture non-linearity. 

This paper aims to compare the performance of various types of machine learning models, namely a random forest, an adaptive boosting model, an artificial neural network, and a long short-term memory neural network. The performance will be compared by forecasting the one day ahead movement of log returns for three different companies. A combination of stock data from 2005 to 2015 and several technical indicators will be used as inputs. We assess the performance of our models on their ability to predict whether or not the log return of a stock moves up or down. Our results are benchmarked using a classical statistical model in forecasting financial data; the Auto Regressive Integrated Moving Average (ARIMA) model.

Empirical results indicate that the long short-term memory neural network model performed with the highest accuracy (achieving 59\%) in forecasting movement of the log returns. The machine learning models consistently outperformed the ARIMA models for all three stocks. The methods and techniques used in this paper are not problem-specifics and can be applied to different recognition and prediction problems.

\end{abstract}
%----------------------------------------------------------------------------------------
%	LIST OF CONTENTS/FIGURES/TABLES PAGES
%----------------------------------------------------------------------------------------

\tableofcontents % Prints the main table of contents


%----------------------------------------------------------------------------------------
%	THESIS CONTENT - CHAPTERS
%----------------------------------------------------------------------------------------

\mainmatter % Begin numeric (1,2,3...) page numbering

\pagestyle{thesis} 

% Chapter 1

\chapter{Introduction} % Main chapter title

\label{Chapter1} % For referencing the chapter elsewhere, use \ref{Chapter1} 

%----------------------------------------------------------------------------------------

% Define some commands to keep the formatting separated from the content 
\newcommand{\keyword}[1]{\textbf{#1}}
\newcommand{\tabhead}[1]{\textbf{#1}}
\newcommand{\code}[1]{\texttt{#1}}
\newcommand{\file}[1]{\texttt{\bfseries#1}}
\newcommand{\option}[1]{\texttt{\itshape#1}}

%----------------------------------------------------------------------------------------


%----------------------------------------------------------------------------------------
%	SECTION 1
%----------------------------------------------------------------------------------------

\section{Introduction To Problem}

Thanks to a plethora of high-profile applications in fields like autonomous transport, intelligent robotics, and image recognition, artificial intelligence (AI) has gained popularity in recent years (Norvig, 2002). Machine learning is a subfield of AI and refers to the general techniques used to extrapolate patterns from large data sets, or as the ability to make predictions on new unseen data based on what is learned (Ahmed \textit{et al.}, 2010). The success of AI is largely due to the use of machine learning methods that can learn and improve over time. This is in contrast to traditional programming, where step-by-step coding instructions are followed (Norvig, 2002). Machine learning methods encompass a variety of models, including support vector machines, Random Forest, and Neural Networks, among others (Ahmed \textit{et al.}, 2010). 

Given the substantial growth and popularity, machine learning has found applications in the field of time series forecasting. Time series data - data points indexed in time -  are found in many problems. One such problem which is widely researched is financial data (Zhang, 2003). Our interest in this domain for this project is stock market data. Forecasting the returns of stocks is a challenging task due to the volatility and non-linearity of the data (Oancea \textit{et al.}, 2014). Classical statistical and econometric methods like auto-regressive integrated moving average (ARIMA) and moving average (MA) have been used for financial time series prediction, but have been found to fail in efficiently handling the uncertain nature of stock market data (Oancea \textit{et al.}, 2014). This is due to the fact that these statistical methods assume the time series is generated from a linear process (Liwicki and Everingham, 2009). Therefore, they achieve poor performance when trying to predict the non-linear movements of stock prices. In recent times, neural networks have become increasingly popular in solving a variety of supervised learning \footnote {Supervised learning is the task of learning a function that maps an input to an output using labelled training data} problems, since they are capable of approximating any non-linear function without any a priori information about the properties of the data (Norvig, 2002).


Recently, neural networks have become popular in solving a variety of scientific and financial problems, because they can approximate any non-linear function without any a priori information about the properties of the data (Norvig, 2002). 

\subsection{Aim of the Paper}

This paper aims to compare the performance of various machine learning models, including two ensemble decision tree methods and two neural networks, in forecasting the movement of log returns of three stocks. More specifically, we compare a Random Forest model, Adaptive Boosting model (AdaBoost), Feed-Forward neural network and a Long Short-Term Memory (LSTM) neural network. We assess these models performance by their ability to predict whether the daily log return of a stock moves up or down, using three measures - accuracy, F1 score and AUC (area under the curve) score. Three stocks are used, namely Google, Motorola and Ford Motors. In addition, we benchmark the performance of our machine learning models to that of a classical statistical approach in forecasting time series data; an auto regressive integrated moving average model (ARIMA).

\subsection{Contributions of Paper}

With the continuous development and growing popularity in neural networks and machine learning, there exists a great deal of literature covering machine learning models in time series forecasting. Several types of neural networks have been used in financial forecasting problems and have been extensively discussed in the literature. However, there have only been a few accounts of comparing neural networks methods and ensemble decision trees methods on financial data. In addition, our paper benchmarks and validates our models by comparing our results with  that of an ARIMA model. 

\section{Theoretical Concepts}

\subsection{Time Series}

From stock market prices to measuring the spread of an epidemic, it is common place for data to be recorded with a time component (Naeini \textit{et al.}, 2010). When the measurements are gathered together, they form a time series. Essentially, a time series is a sequence of observations in chronological order, like the daily log returns on a stock  (Brillinger, 2001). The data do not have to be necessarily equally spaced, however this is a common assumption made (Liwicki and Everingham, 2009). For example, daily log returns on a stock may only be available for weekdays, with additional gaps on holidays. For simplicity, in this paper we regard the consecutive observations as equally spaced. 


Time-series analysis serves as a fundamental statistical method for analysing the behaviour of time-dependent data and for forecasting. In the financial domain, time series modelling is considered the traditional technique to model financial data and is still widely used in finance today (Brillinger, 2001). The time series model chosen to benchmark the performance of our non-linear machine learning methods, is the ARIMA model. The ARIMA is a univariate time series model, which is widely used for forecasting time series using the series’ past values (Brillinger, 2001). 

The choice of using stock market data is motivated by the inherit non-linearity \footnote{Non-linear time series are generated by nonlinear dynamic equations. They display features that cannot be modelled by linear processes: time-changing variance, asymmetric cycles, higher-moment structures, thresholds and breaks.} of the time series data. Because of the market's unpredictability, which stems from constantly changing economic and political situations, as well as the stock market's instability, it is difficult to accurately anticipate stock market movement only by looking at price history (Shively, 2003). The complex dynamics of the data serves as an interesting foundation in which we can assess and compare the accuracy of each machine learning model. 

%-----------------------------------
%	SUBSECTION 2
%-----------------------------------

\subsection{Forecasting}

The process of forecasting can be described as predicting the future using past and current data (Brillinger, 2001). Forecasting spans many areas including - but not limited to - business, health science, environmental science and finance (Lago \textit{et al.}, 2018). It is an indispensable tool that helps, for example, financial institutions manage the uncertainty in the future for investments. Forecasting is used by businesses to make operational choices in areas such as buying, marketing, and advertising. Hence, a great deal of research has been conducted in identifying and improving forecasting models and techniques. 

Forecasting models can be linear \footnote{A linear time series is one where, for each data point $X_t$, that data point can be viewed as a linear combination of past or future values or differences.} as well as non-linear. The machine learning methods used in this paper are examples of non-linear forecasting models. 


Although linear models have shown to be effective tools for forecasting, they are intrinsically constrained when dealing with non-linear data. As a consequence, forecasts as well as other conclusions drawn from them could be misleading when applying these linear models on non-linear data. This had led to great interest in non-linear time series models in recent decades. Models such as the threshold autoregressive models (Tong, 1990) and the exponential autoregressive model (Haggan and Ozaki, 1981), are examples of non-linear models that have been developed to handle the non-linearities in the data of concern. However, one key issue that arises when using these non-linear models instead of linear models is that there is no unified theory that can be applied to all non-linear models. This is due to the fact that they need the imposition of assumptions about the specific form of non-linearity. Moreover, the pre-specified non-linear model may not be broad enough to capture all essential characteristics. One alternative way to deal with non-linearities in data is to use non-linear machine modelling approaches (Kuo and Huang, 2020). In contrast to the above model-based methods, non-linear methods like machine learning models are data driven and are thus capable of capturing complex patterns inherent in the data without needing any a priori information about the data (Liwicki and Everingham, 2009).

%-----------------------------------
%	SUBSECTION 2
%-----------------------------------

\subsection{Neural Networks}

Neural networks are a class of machine learning algorithms. They are made up of a collection of neurons that connect through various layers. Neural networks attempt to learn the mapping of the input data to the output data, on being provided with a training set. These models are non-linear and operate without prior beliefs about the functional forms of the data.

The objective of machine learning methods is the same as that of traditional linear statistical methods. Both machine learning and traditional linear techniques in statistics aim to improve forecasting ability, by minimising some loss function (Kuo and Huang, 2020). Their difference lies in how such minimisation is done. Machine learning methods utilise non-linear algorithms to do so while statistical methods employ linear processes (Kuo and Huang, 2020).  

The application of neural networks in prediction problems is both pertinent and promising due to their special characteristic - being general function approximators (Liwicki and Everingham, 2009). Neural networks learn from examples and can capture the subtle functional relationship existing between the input and output data (Kuo and Huang, 2020).  A collection of trainable parameters called weights are dispersed over multiple layers to achieve the mapping. The weights are learned by the backpropagation algorithm whose aim is to minimise a loss function. Given their generalisation ability, after training a neural network they are able to recognise new patterns from unseen data, even if the unseen data contain noisy information, at least in principle (Liwicki and Everingham, 2009). This follows from the universal approximation theorem (Agrawal \textit{et al.}, 2013). The theorem states that any neural network can approximate any complex continuous function. We can learn about any relationship between the input and the output of the system, thus enabling us to learn any complicated relationship between the input and the output of the system (Kuo and Huang, 2020). We look at implementing two types of neural networks in this paper - a feed-forward and long short-term memory neural network.


%-----------------------------------
%	SUBSECTION 2
%-----------------------------------

\subsection{Ensemble Decision Trees}

Decision trees are a form of supervised machine learning algorithms which employ an if-then-else decision methodology to fit data and forecast results (Ampomah \textit{et al.}, 2020). Ensemble decision tree methods are those which combines several decision trees to produce better predictive performance than by using a single decision tree (Ampomah \textit{et al.}, 2020). The main principle behind the ensemble model is that a group of weak learners come together to form a strong learner \footnote{A learner refers to a model. A weak learner is a model that has a low skill, meaning that its performance is slightly above a random classifier for binary classification or predicting the mean value for regression. In contrast a strong learner is one that performs well on a predictive modelling problem}. Random Forests and Adaptive Boosting models are built on this same principle (Nti \textit{et al.}, 2020).  A Random Forest algorithm entails building a multitude of decision trees and then merging them together to obtain a more accurate prediction (Ampomah \textit{et al.}, 2020). Adaptive boosting, or AdaBoost, is a type of boosting technique, which involves using very short decision trees as weak learners that are added sequentially to the ensemble (Ampomah \textit{et al.}, 2020). The subsequent models attempt to correct the predictions made by the model before it in the sequence. The Random Forest and AdaBoost algorithms represent some of the most popular ensemble methods (Nti \textit{et al.}, 2020). 


\subsection{Structure of Paper}

This paper consist of 7 chapters and the remaining chapters are organised as follows: chapter 2 presents a review of the literature published on predicting stock market data, neural networks and exploration into the performance of machine learning methods in comparison to statistical ones, with respect to applications in financial data prediction. The next chapter, then briefly explores the data sets for the stocks and try to understand the data at hand through exploratory data analysis. Following this, chapter 4 describes the methodology of the paper. Here we have a look at the data and theory of the various neural network models used for our analysis, the experiments performed, and the training algorithms used. In chapter 5, the model specification and optimisation processes for our experiment are provided. Chapter 6 presents the results of our analysis. Discussion and concludion then follow in chapter 7. 







 % Chapter Template

\chapter{Literature Review} % Main chapter title

\label{Chapter2} % Change X to a consecutive number; for referencing this chapter elsewhere, use \ref{ChapterX}


Due to the face that temporal data is created in a variety of circumstances, time series forecasting is a very common data modelling challenge. The financial domain has extensive research into modelling and analysis of time series data. Some of the early approaches to financial forecasting were based upon autoregressive models such as ARIMA (Binner \textit{et al.}, 2005). Such models produce a forecast estimate which is dependent on a linear combination of past values and errors. If the assumption of stationarity is true and the series is generated by a linear process, autoregressive models have been found to perform well (Binner \textit{et al.}, 2005).  The data we are concerned with are stock returns which are generated from a non-linear process (Shivley, 2003). Machine learning models overcomes the shortcoming of linear modelling processes as they makes no assumptions about the prior distribution of the data (Binner \textit{et al.}, 2005). 

In this section we cover a variety of research on financial time series including a comparison on the literature between linear time series and non-linear machine learning methods in forecasting. We begin this review of the relevant literature, by looking at the random walk and efficient market hypothesis. 

%----------------------------------------------------------------------------------------
%	SECTION 1
%----------------------------------------------------------------------------------------

\section{Random Walk and Efficient Market Hypothesis}

It was widely believed, for many years, that stock prices could not be predicted. The random walk theory and the efficient market hypothesis (EMH) gave rise to this notion (Malkiel, 1973).  The Random-walk theory (see Appendix A) states that the price of a stock swings irrespective of its historical performance. That is, stock prices will move in a random walk pattern, and that any prediction of stock movement will be around 50\% accurate. According to EMH, an asset's current price always reflects all previously available information in real time. Hence, it cannot be predicted to consistently earn economic gains greater than the overall market average. As such, the use of prediction algorithms to determine future trends in stock market prices contradicts these two notions.

After exposure to the literature, there appears to exist contention surrounding financial markets and in particular the random walk hypothesis and EMH. Many researchers accept the random walk hypothesis, including Biondo \textit{et al} (2013) who conducted a paper which supported the theory. In studying the performance of technical trading methods, they found that a random strategy outperformed some of the more traditional technical trading methods, such as Moving Average Convergence Divergence (MACD) and Relative Strength Index (RSI), in a series of tests.  Other academics, contend that stock values can be forecasted, at least to some extent (MacKinlay, 1999).
 That is, some researchers have claimed that movements in market prices are not random. Rather, they behave in a highly non-linear, dynamic manner (MacKinlay, 1999). Papers published by Smith (2003), Nofsinger (2005), and Bollen \textit{et al.} (2011), provide. evidence contrary to the notions suggested by EMG and the random. walk. hypothesis.  These studies all conclude that that the stock market can be predicted to some degree and therefore. This is in contrast to and undermines the EMH’s underlying assumptions. In this paper, we are not concerned with proving or disproving the predictability of markets. We are simply comparing models using stock data, however the research mentioned provided some context to the data domain. 

%----------------------------------------------------------------------------------------
%	SECTION 2
%----------------------------------------------------------------------------------------
\section{Stock Market Prediction Strategy}

The field of financial forecasting is characterised by noise, non-stationarity, lack of structure and high degree of uncertainty. There are many factors inherent in finance that have an effect on its dynamics and movements, such as political events and traders’ expectations. Aside from the discord surrounding predicting stock markets, there has been a great deal of research on the predictability of stock returns. Current strategies regarding financial investing are broadly classified into two methods; fundamental and technical analysis. 

Fundamental analysis refers to the type of trading analysis which involves the in-depth analysis of a company’s performance and the profitability. This is done so as to measure a company's intrinsic value. Fundamental analysis studies the company in terms of its product sales, man power quality, infrastructure, profitability on investment (Agrawal \textit{et al}, 2013). As such, the stock value of a company is estimated by analysing its margins, earnings, return on equity and profits amongst others as well as other economic factors. It based its principles that the market value of a stock tends to move towards its “intrinsic value”  (Agrawal \textit{et al}, 2013). 

In contrast, technical analysis is an method for forecasting the direction of prices. This is done, not by trying to measure a ciompany's intrinsic value, but by through the study of past market data, primarily price and volume, mostly to trigger the buy or sell rules in the technical analysis (Kirkpatrick II and Dahlquist, 2010). It looks for peaks, bottoms, trends, patterns, and other factors affecting a stock's price movement.

Technical analysis is based on the following assumptions:
\begin{itemize}
\item Prices are solely determined by the supply demand relationship;
\item Prices fluctuate in response to trends;
\item Changes in supply and demand lead trends to reverse;
\item Changes in supply and demand can be graphed;
\item Patterns on charts seem to repeat themselves (Kirkpatrick II and Dahlquist, 2010)
\end{itemize}

Given the above, it is clear that technical analysis methodology ignores external aspects such as political, social, and macroeconomic factors (Cambell, 2012). A paper published in 2004, by Wong \textit{et al} (2010), looked at the role of technical analysis in signalling the timing of stock market entry and exit. Using Singapore data, their results indicated that the indicators can be used to generate positive return from the Singapore Stock Exchange (SES). Moreover, Gao (2018) looked at improving improving stock closing price prediction using recurrent neural network and technical indicators. Gao (2018) employed an LSTM model coupled with technical indicators, to predict the closing price of Apple stock. They compared their model to various other forecasting models, including an ARMA, feed-froward neural network and support vector machine. Their proposed LSTM model with technical indicators outperformed all the other models. 

%----------------------------------------------------------------------------------------
%	SECTION 3
%----------------------------------------------------------------------------------------
\section{Neural Networks in Financial Forecasting}

Neural networks possess certain characteristics, which deliver very promising potential in prediction problems. Neural networks are particularly suited to settings in which identifying links between data is challenging but there is a large enough training data available. Due to their innate ability to generalise and approximate any complex continuous function, they are able to learn any relationship between a system's input and output  (Binner \textit{et al.}, 2005). This makes them extremely adept in prediction problems, and hence there exists vast research available on using this machine learning architecture on the financial data  (Binner \textit{et al.}, 2005).

The earliest application of neural networks in forecasting was in 1964 by Hu (Chen \textit{et al.}, 2015). Although it did not receive significant attention due to the absence of any learning method for neural networks - thus, making it impossible to apply these networks to complex precision problems. However, in response to the work on error backpropagation learning, which was published by Werbose (1982) and by Rumelhart \textit{et al} (1986), the interest in neural networks have grown rapidly since.

In 1988, White (1988) made the first attempt to model a financial time series using a neural network model. Looking at asset data for IBM, White (1988) tried to model a neural network for decoding the non-linear regularities in IBM's asset price movement. His work, albeit. limited in scope, helped in establishing evidences against EMH (Chen \textit{et al.}, 2015). Moreover, in 1990, Kimoto \text{et al.} (1990), used neural networks to predict the index of the Tokyo stock market, and achieved precision of 63\% precision (Naeini, 2010).

Recurrent neural networks (RNNs) are a class of artificial neural network. In these networks, connections between neurons form a directed cycle. RNNs can be compared to normal feed-forward networks, but with loops. As such, they naturally exhibit dynamic temporal behaviour for a time sequence. This has led to them being studied in the context of time series forecasting extensively. Additionally, RNNs have produced better results with respect to time series problems than other machine learning methods (Chen \textit{et al.}, 2015). More specifically, in this paper, we look at using a long short term memory (LSTM) network, which is a type of RNN model LSTMs were introduced in 1997 by Hochreiter and Schmidhuber, to address certain limitations that standard RNNs had (Chen \textit{et al.}, 2015). 

When compared to other machine learning modelling methods, these networks have often produced some of the best results in the context of sequential data (Graves, 2012). Their impressive performance is particularly renowned in the field of Natural Language Processing (NLP), and in handwriting recognition (Liwicki, 2009). The promising results of LSTMs models in their applications have triggered their use in the financial forecasting. 

From the past literature, LSTM have delivered promising results in forecasting financial assets. A paper published by Chen \textit{et a.l} (2015), used price data to forecast price movement. More, specifically he used historic price data and technical indicators in addiction to stock indexes to predict if a stock’s price would increase or decrease on a given day. Their research looked at different stocks from the Brazilian Stock Exchange. Chen \textit{et a.l} (2015) were able to achieve promising results from their analysis. They obtained an average of 55.9\% of accuracy when predicting if the price of a particular stock is going to go up or not - using their LSTM model. 

%----------------------------------------------------------------------------------------
%	SECTION 4
%----------------------------------------------------------------------------------------


\section{Ensemble Decision Trees in Financial Forecasting}

Due to the success of machine learning algorithms in a variety of fields since their inception, ensemble methods have also risen in popularity. Ensemble methods is a machine learning technique that combines several base models in order to produce one optimal predictive model. Fast algorithms such as decision trees are commonly employed in ensemble methods, however ensemble methods do not only pertain to decision trees.  For the purposes of this study we shall be implementing ensemble decision trees, more specifically a Random Forest and an Adaptive Boosting model. 

Recently there have been several publications regarding the evaluation of tree based ensemble machine learning models in financial forecasting. Ampomah \textit{et al.} (2020) published a paper comparing the effectiveness of several tree based ensemble models. Their paper included a random forest, XGboost model, and even an Adaptive Boosting model. The authors assessed the models on their ability in forecasting the direction of stock price movement, using several performance metrics such as, accuracy, precision and area under receiver operating characteristics curve (AUC ROC). Ampomah \textit{et al.} (2020) found that when using their training set, the AdaBoost model performed the best, while on the test. set, the Extra Trees classifier outperformed the other models. In the study the Adaptive Boosting model was the second best model, followed closely by the Random Forest model.

Moreover, a study by Nti \textit{et al.} (20), compared ensemble learning methods for classification and regression machine learning tasks in stock market prediction. The goal of the paper was to clarify which ensemble strategies are most suited for stock market prediction problems. In the study, a host of methods were used; namely boosting, bagging, blending and stacking. The models used included decision trees, support vector machines and neural networks. In their comparative analysis, they found that stacking and blending ensemble techniques offered higher prediction accuracies compared with bagging and boosting. 

%----------------------------------------------------------------------------------------
%	SECTION 4
%----------------------------------------------------------------------------------------

\section{Comparison of Linear and Non-Linear Forecasting Methods}

The literature argues that neural networks offer significant theoretical advantages over traditional statistical methods. Neural networks have been mathematically shown to be universal approximators of functions (Hornik \textit{et al.}, 1989). Hence, neural networks can - in theory - approximate whatever functional form best characterises a time series. Neural networks are also inherently non-linear and so can estimate nonlinear functions considerably better than linear methods (Rumelhart and McClelland 1986). However, empirically their application and results against traditional methods have been long contested. Several studies have compared neural networks and traditional time series approaches. Most of these studies have used the data from the M-competition (Makridakis \textit{et al.}, 1982), which includes 1001 real time series (Ahmed. \textit{et al.}, 2010). 

Looking at the original M-competition study, various groups of forecasters were given all but the most recent data points in each series. Each group consisted of forecasters who were regarded as experts in a particular technique. The groups were free to apply any approaches in. their. domain of competence to forecast the time series. After each group prepared its forecasts, Makridakis compared the forecasts to the actual values. The specific results of this competition are reported in Makridakis et al. (1982). One of the main conclusions from the competition, were that statistically sophisticated or complex methods do not necessarily produce more accurate forecasts than simpler ones. This inspired our decision to include a simple benchmark model to compare the performance our machine learning models. 

In the years that followed the competition ,the data which contained all 1001 series have been made available. This sparked interest in the community and led to many studies using the data and empirically analysing the conclusions from the M-competition. The results of these studies have contrasted the findings of the competition, by demonstrating that neural network models in some cases appear to outperform simpler methods.  Sharda and Patil (1990) made use of. a subset of the competition data. Their findings suggested that neural network models performed as well as the automatic Box-Jenkins (Auto-box) procedure. Similarly, Tang \textit{et al.} (1990), found that neural network models and Box-Jenkins models produce comparable results, when used on time series with a long history. In 1991 however, Foster \textit{et al.} (1991)  found that neural networks to be inferior to Holt's, Brown's, and the least squares statistical models for time series of yearly data. When comparing the models on quarterly data, they found the models to be comparable, using the competition data. Kang (1991), compared neural networks and Auto-box on the 50 M-competition series and found that it was the most appropriate technique for forecasting. An important finding from Kang's research was that the neural networks often performed better when predicting on a forecasting horizon a few periods ahead. This effect has also been noted when using neural networks to forecast macroeconomic data (White, 1994) and currency exchange rates (Zimmerman, 1994). Finally, Tim Hill \textit{et al} (1996), conducted research on time series forecasts produced by neural networks and compared them with forecasts from six statistical time series methods generated in the M-competition (Makridakis \textit{et al.}, 1982). Across monthly and quarterly time series, the neural networks did significantly better than traditional methods. However, the neural network model and traditional models were comparable on the annual data. 

More recently, a paper published by Binner \textit{et al.} (2005), compared neural networks to traditional time series models; univariate autoregressive integrated moving average (ARIMA) and multivariate vector autoregressive (VAR) models. They did not use the competition data, however they found that the best models for their neural network outperform the traditionally used linear ARIMA and VAR models in macroeconomic forecasting.This is in line with findings from White (1994) and Zimmerman (1994). The authors attributed the gain in forecasting accuracy for their neural network models, to their capability to capture nonlinear relationships between macroeconomic variables (Binner \textit{et al.}, 2005). 

The results from the literature above are encouraging but equivocal; we were thus inspired to include a benchmark model to our comparative analysis of machine learning models - namely, an ARIMA model which, in fact, outperformed neural network models in the M-competition (Makridakis \textit{et al.}, 1982). 


%----------------------------------------------------------------------------------------
%	SECTION 4
%----------------------------------------------------------------------------------------

\section{Comparison of Machine Learning Forecasting Methods}

Although there has been extensive research into the applications of individual machine learning methods for forecasting financial data, there does not exist as much research into comparisons of these existing methods. One of the few, was published by Ahmed NK \textit{et al.} (2010). Their paper compared empirically the performance of several families of machine learning models for time series forecasting. More specifically, they compared in their study the performance of eight machine learning models with respect to their accuracy applied to a subset (1045 series) of the M3 competition data. The M3 competition is a sequel of M forecasting competitions, organised by Makridakis and Hibon (2000). The data builds on its predecessors and consists of 3003 business-type time series, covering various fields. In the study, they consider only the monthly time series. 

Before computing the forecasts, they pre-processed the series in order to achieve stationarity in their mean and variance. Ahmed NK \textit{et al.} (2010) calculated one-step-ahead forecasts for each one of the time series. To summarise the findings of the paper, the family of models which ranked the best turned out to be a multilayer perceptron, followed closely by a Gaussian processes and Bayesian neural network. The worst performing family of machine learning models was the radial basis functions. Our study extends this analysis, by looking at four different machine learning models, assessed on financial stock data.
 
 % Chapter Template

\chapter{Exploratory Data Analysis} % Main chapter title

\label{Chapter3} % Change X to a consecutive number; for referencing this chapter elsewhere, use \ref{ChapterX}

This section looks at the implementation of various tools to help decompose and analyse the structure of the datasets. An important part of statistics, involves the analysis, organisation, presentation, and summary of data. 

%----------------------------------------------------------------------------------------
%	SECTION 1
%----------------------------------------------------------------------------------------

\section{Time Series Analysis}

A time series is a sequence of observations in chronological order and forecasting is predicting the future using past data.  In our project we implement several models for time series forecasting. The data we are concerned with are the log returns of three stocks. It is not often that the original data will be appropriate to use immediately for processing. Figure 3.1 looks at the evolution of the closing price for the three stocks - Ford Motors, Google and Motorola. 

\begin{figure}[h]
\centering
  \includegraphics[scale =0.34]{/Users/pavansingh/Google Drive (UCT)/STA Honours/Project/Thesis/Python Coding/Figures/EDA/GrowthOfStocks.pdf}
  \caption{Closing price for Google, Ford and Motorola stocks between 2005 and 2015.}
  \label{}
\end{figure}

Figure 3.1 illustrates the tremendous growth Google has had during the period 2005 to 2015. In contrast, Motorola and Ford stock value has not experienced similar growth. In fact Ford stock value appears to have faced extended periods of negative growth between 2006 and 2010. The decrease of stock value between 2008 and 2010 is attributed to the Global financial crisis. Ford Motors stock value is considerably lower than Google, and hence their evolution of price is not clearly illustrated in the plot. A clearer representation for the evolution of each stock value is found in appendix B. 

Given time series data, it is necessary to pre-analyse, as well as preprocess the time series. It is necessary to remove any known characteristics of the data which could hamper the performance of the models' forecast accuracy (Metcalfe, 2009). For example, figure 3.1 illustrates time series data with trend \footnote{Trend refers to the phenomenon that the average value of sequence elements is constantly rising or falling.}. Trend and seasonality effects, mean that the behaviour of a time series is subject to repetition and change at the same time. While similar patters may repeat over time - such as periodic patterns due to a periodic influencing factor (e.g. day of the week or time of year for trading), the frequency and intensity of those are usually not constant. Another reason for eliminating trends and seasonalities (or, for that matter, any other clearly visible or well-known pattern) is that many traditional techniques such as ARIMA, require stationarity of the time series. 

In our project we are concerned with predicting the movement of daily log returns, as such we computed the daily log returns for the three stocks. Figure 3.2 (pictured below) illustrates the daily log returns for Ford motors over the period 2005 and 2015. The daily log returns for Google and Motorola can be found in appendix B.

\begin{figure}[h]
\centering
  \includegraphics[scale =0.34]{/Users/pavansingh/Google Drive (UCT)/STA Honours/Project/Thesis/Python Coding/Figures/EDA/LogReturnsFordMotors.pdf}
  \caption{Log Daily Returns for Ford Motors stock from 2005 to 2015}
  \label{}
\end{figure}

The data shown in Figure 3.2 are typical of daily log return data for stocks. The mean of the series appears to be stable with an average return of approximately zero. As such, despite the individual large random fluctuations in daily log returns for Ford Motors, the series appears stationary, meaning that the nature of its random variation is constant over time. We also see volatility clustering, because there are periods of higher, and of lower, variation within each series that tend to be clustered together. The volatility clustering is not indicative of a lack of stationarity but rather can be viewed as a type of dependence in the conditional variance of each series (Metcalfe, 2009). 

\section{Stationarity}

Stationarity is a property of a time series. A stationary series is one where the value of the series are not a function of time. That is, the statistical properties of the series like mean, variance and autocorrelation are constant over time (Metcalfe, 2009). It is important for the data to be stationary in order to avoid spurious regression - a regression that produces misleading statistical evidence of relationships between independent non-stationary variables. In order to receive consistent and reliable results, non-stationary data needs to be transformed into stationary data. Non-stationary data may be generated by an underlying process that is affected by a trend, a seasonal effect, presence of a unit root, or a combination of all three. 

By computing the log returns, from visual inspection (shown in Figure 3.2) the time series appears to exhibit stationarity. However visual inspection is not satisfactory. To quantitatively determine if a given series is stationary or not, we used the Augmented Dickey Fuller test (ADF Test). Here, the null hypothesis is the time series possesses a unit root and is non-stationary (Metcalfe, 2009). The results of the test are shown below in Table 3.1. 

\begin{table}[h]
\centering
\begin{tabular}{llll}
\toprule

               & Ford Motors & Google    & Motorola  \\
               \midrule

Test-Statistic & -8.564      & -17.097   & -8.920    \\
p-value        & 8.556e-14   & 7.522e-30 & 1.044e-14 \\
\bottomrule

\end{tabular}
\caption{Results from augmented Dickey Fuller test.}
\end{table}

Across all stocks, we observe low p-values, hence we reject the null hypothesis and can proceed assuming our series are stationary. We look to achieve stationarity since it is  necessary so that averaging lagged products over time will be a sensible. Moreover, the sample autocorrelation  function becomes consistent - so we able to estimate autocorrelations with precision. And additionally, it is required for our ARIMA modelling. Our log transformation of the data, proved suitable to create a stationary series. 


%-----------------------------------
%	SUBSECTION 1
%-----------------------------------
\section{Data and Variables}

The variables in the dataset used in this project is made up of OHLCV (open, high, low, close, volume) as well as several technical indicators. The kernel density estimation is a non-parametric way to estimate the probability density function of a random variable. To illustrate the distribution of log daily returns for Ford Motors, we use both a histogram and kernel density estimator.

\begin{figure}[h]
    \centering
    \subfloat{{\includegraphics[scale=0.43]{/Users/pavansingh/Google Drive (UCT)/STA Honours/Project/Thesis/Python Coding/Figures/EDA/Ford_Hist.pdf} }}
    \qquad
    \subfloat{{\includegraphics[scale=0.19]{/Users/pavansingh/Google Drive (UCT)/STA Honours/Project/Thesis/Python Coding/Figures/EDA/Ford_KDE.pdf} }}%
    \caption{Histogram and kernel density estimator.}%
    \label{}%
\end{figure}

Visually the outliers may be difficult to see due to the large sample size. We see that the returns closely resemble a symmetric distribution. However, we can see that the distribution has rather narrow peak to be normally distributed. This may suggest later exploring if the distribution is in fact normal. We can look at a QQ-plot to show the distribution of the daily log returns data against the expected normal distribution.


\begin{figure}[h]
\centering
  \includegraphics[scale =0.3]{/Users/pavansingh/Google Drive (UCT)/STA Honours/Project/Thesis/Python Coding/Figures/EDA/QQplotFord.pdf}
  \caption{QQ-Plot of daily log returns for Ford Motors.}
  \label{}
\end{figure}

If the set of quantiles from the Ford data come from a normal distribution, then we should see the points forming a line that’s roughly straight (Metcalfe, 2009). Figure 3.4 shows that this is not the case. We notice the points fall along a line in the middle of the graph, but curve off in the extremities. That is, the Q-Q plot highlights the fat tails of the distribution with extreme values more frequent than the normal distribution would suggest 
Normal Q-Q plots that exhibit this behaviour usually indicate that the data have more extreme values than would be expected if they truly came from a Normal distribution. Similarly, the daily log returns for Motorola and Google also display departures from normality’s in the tails - see appendix B. We further explore this in the next section 3.4.

We look at some of the technical indicators included in the study on Ford Motors. The indicators are overlayed on the closing price for Ford closing price, 200 days prior to our testing set - the unseen data on which we assess our models.

\begin{figure}[!h]
\centering
  \includegraphics[scale =0.34]{/Users/pavansingh/Google Drive (UCT)/STA Honours/Project/Thesis/Python Coding/Figures/EDA/TechnicalIndicators.pdf}
  \caption{Several technical indicators on closing price of Ford Motors stock taken from the last 200 Days.}
  \label{}
\end{figure}

The Bollinger Bands are envelopes plotted at a standard deviation level above and below a simple moving average of the price. These help determine whether prices are high or low on a relative basis. We see there are a couple cases where the closing price deviates out of the envelope in the past 200 days. When stock prices continually touch the upper Bollinger Bands the prices are thought to be overbought; conversely, when they continually touch the lower band, prices are thought to be oversold, triggering a buy signal. A clear case of this signal working well, is during the last few days, where the closing price touched the lower band. Days later the stock closing price increased.  These bands are used in pairs, both upper and lower bands and in conjunction with a moving average. The Moving Average essentially takes a specified number of past days, takes the average of those days, and plots it on the graph. For a 7-day moving average (MA7 on figure 3.5), it takes the last 7 days, and similarly, the 21-day moving average using the past 21 days prices. These short term moving averages are best suited for short-term trends and trading (Nelson \textit{et al.}, 2017). The moving averages help with identifying trends to aid in identifying buy and sell signals. When the closing price breaks past an upwardly sloping moving average, this could indicate it is a good time to purchase the stock. The moving average can also act as a support line, that suggests that a closing price which approaches a support line may rally up again and thus signal a buy. 

We also notice that the Ford Motors stock value has experienced substantial change in value over this 200 day period, reaching a high value of 12.77 and a low of 9.81. 

%-----------------------------------
%	SUBSECTION 2
%-----------------------------------

\section{Summary Statistics}

By computing the summary statistics for our data, we can quickly understand the characteristics of our data sets. Table 4.2 shows the summary statistics for all three stocks.

\begin{table}[h]
\centering
\begin{tabular}{lrrr}
\toprule
{} &  Ford Motors &       Google &     Motorola \\
\midrule
count &  2515 &  2515 &  2515 \\
mean  &     0.000021 &     0.000658 &    -0.000008 \\
SD   &     0.029800 &     0.019708 &     0.023202 \\
min   &    -0.287682 &    -0.123402 &    -0.207639 \\
25\%   &    -0.012454 &    -0.008286 &    -0.009973 \\
50\%   &     0.000000 &     0.000384 &     0.000150 \\
75\%   &     0.012354 &     0.010165 &     0.009720 \\
max   &     0.258650 &     0.182251 &     0.174097 \\
kurtosis   &     13.1942306 &     9.106027 &     9.678477 \\
skewness   &     0.0253992 &     0.360065 &     -0.500873 \\
\bottomrule
\end{tabular}
\caption{Summary statistics for Ford, Google and Motorola stocks from 2005 to 2015. Where SD refers to standard deviation and 25\%, 50\%, 75\% refer to the lower, median and upper quartiles respectively.}
\end{table}

In table 3.2, the mean daily log return for the three stocks are very small and close to 0. The range of returns appears to be quite high across all three stocks. The sample departures from the normal distribution are summarised by the coefficients of skewness and Kurtosis. Formally, kurtosis is a measure of the combined weight of a distribution's tails relative to the centre of the distribution, while skewness is the measure of how much the probability distribution of a random variable deviates from the normal distribution (the skewness for a normal distribution is zero). If the kurtosis is greater than 3, then the dataset has heavier tails than a normal distribution (more in the tails). We see that Ford motors has the greatest kurtosis value. Google and Motorola have similar values for Kurtosis.  In the previous section we noted that the distribution of the daily log returns for Ford Motors greatly departs from normality in the tails. We can further explore the tails of the distribution for daily log returns of all three stocks. This is illustrated by the box-plot in figure 3.6.

\begin{figure}[h]
\centering
  \includegraphics[scale =0.3]{/Users/pavansingh/Google Drive (UCT)/STA Honours/Project/Thesis/Python Coding/Figures/EDA/BoxPlotLogReturns.pdf}
  \caption{Box-Plot of Daily Log Returns for Ford, Google and Motorola stock from 2005 to 2015}
  \label{}
\end{figure}

We again notice that the mean is close to 0 amongst all the stocks and spread is generally small for the central part of the distribution - this indicates that the data are very peaked but long-tailed. Ford Motors daily log returns appears to have the greatest variation amongst the stocks.  We notice a large proportion of extreme values for all the stocks. Although Ford Motors, has a noticeably greater range of values. 

Finally, we can assess the association between the three chosen stocks. We explore a pairs plot, which is simply a matrix of scatterplots.

\begin{figure}[h]
\centering
  \includegraphics[scale =0.51]{/Users/pavansingh/Google Drive (UCT)/STA Honours/Project/Thesis/Python Coding/Figures/EDA/PaiorwisePlot.pdf}
  \caption{Pairs Plot of Log Daily Returns for Ford, Google and Motorola}
  \label{}
\end{figure}

Figure 3.7, lets us see both distribution of single variables and relationships between two variables. We note that Ford and Google exhibit a stronger relationship than Motorola, where higher returns for Google stock are associated with higher log returns of Ford Stock in general. These findings are confirmed by computing the correlation between these variables. The higher correlation value between Google and Ford indicates that the returns of the two time series data tend to move together.

\begin{figure}[h]
\centering
  \includegraphics[scale =0.25]{/Users/pavansingh/Google Drive (UCT)/STA Honours/Project/Thesis/Python Coding/Figures/EDA/StockHeatMap.pdf}
  \caption{Correlation Heat Map of Log Daily Returns for Ford, Google and Motorola stocks for 2005 to 2015}
  \label{}
\end{figure}



%----------------------------------------------------------------------------------------
%	SECTION 2
%----------------------------------------------------------------------------------------

\section{Autocorrelation}
After our time series has become stationary by transformation, the next step for fitting our ARIMA model is to determine whether AR or MA terms are needed to model any autocorrelation that remains in the transformed series. By looking at the autocorrelation function (ACF) and partial autocorrelation (PACF) plots of the series, we can tentatively identify the values of $p$ (AR) and $q$ (MA) terms that are needed for our ARIMA model. Autocorrelation is the correlation of a series with its own lags. The autocorrelation function )ACF) plot in figure 3.9 is shown below with the confidence band. The Partial Autocorrelation Function (PACF) conveys similar information but it conveys the pure correlation of a series and its lag, excluding the correlation contributions from the intermediate lags. We use the plots of these two measures to infer the parameterisation of our ARIMA model. 

\begin{figure}[h]
    \centering
    \subfloat{{\includegraphics[scale=0.205]{/Users/pavansingh/Google Drive (UCT)/STA Honours/Project/Thesis/Python Coding/Figures/EDA/ACF_ford.pdf} }}
    \qquad
    \subfloat{{\includegraphics[scale=0.205]{/Users/pavansingh/Google Drive (UCT)/STA Honours/Project/Thesis/Python Coding/Figures/EDA/PACF_ford.pdf} }}%
    \caption{Autocorrelations and Partial Autocorrelation Functions for Log Returns of Ford Motors}
    \label{}%
\end{figure}

The blue shaded region in figure 3.9 is the confidence envelope with default value of $\alpha$ = 0.05. Anything within this range represents a value that has no significant correlation. The ACF and the PACF show similar patterns with autocorrelation at several lags appearing significant. The results of figure 3.9, indicate that returns may be autocorrelated till lag 2. We also note that lags 10, 13, 19 and 20 appear significant. The lags that appear to be significant, can be used to parameterise our ARIMA model. The order $q$ of the MA process of our ARIMA model is obtained from the ACF plot. We note that the PACF in figure 3.9 also has very few significant spikes. We use this figure to infer suitable parametrisation for the $p$ in our ARIMA model. We manually configured ARIMA models for different combination of values for $p$ and $q$ to achieve the best model. We chose to try values of $p = 0,1,2$ and $q = 0,1,2$. The length of the lag $p$ is to be chosen so that the residuals are not serially correlated. For examining the information criteria for choosing lags, we went about looking to minimise the Akaike information criterion (AIC). The PACF and ACF plots of the other stocks can be seen in appendix B.






 \chapter{Methodology} % Main chapter title
\label{Chapter4}





\section{Modelling Approach}
\label{sec:Modelling Approach}

\subsection{Formulating the Problem}
\label{subsec:Formulating the Problem}

\subsection{Implementation and Approach}
\label{subsec:Implementation and Approach}























\section{Neural Collaborative Filtering}
\label{sec:Neural Collaborative Filtering}

\subsection{Background}
\label{subsec:Background}

\subsection{Algorithm Formulation}
\label{subsec:Algorithm Formulation for NCF}

\subsubsection{User and Item Embeddings}
\label{subsubsec:User and Item Embeddings}

\subsubsection{Neural Network Architecture}
\label{subsubsec:Neural Network Architecture}

\subsubsection{Loss Function}
\label{subsubsec:Loss Function}


\subsection{Algorithm Summary}
\label{subsec:Algorithm Summary}

\subsubsection{Model Training}
\label{subsubsec:Model Training}

\subsubsection{Hyperparameter Tuning}
\label{subsubsec:Hyperparameter Tuning}

\subsection{Incorporating Text and Sentiments}
\label{subsec:Incorporating Text and Sentiments}

\subsubsection{User Review Text}
\label{subsubsec:User Review Text}

\subsubsection{User Review Sentiments}
\label{subsubsec:User Review Sentiments}


\subsection{Training Results}
\label{subsec:Training Results}

\subsection{Final Model Specifications}
\label{subsec:Final Model Specifications}

\subsection{Summary}
\label{sec:Summary for NCF}

\section{Benchmark Models}
\label{sec:Benchmark Models}

\subsection{Non-Negative Matrix Factorisation}
\label{subsec:Non-Negative Matrix Factorisation}

\subsubsection{Algorithm Formulation}
\label{subsubsec:Algorithm Formulation}

\subsubsection{Adjusting for Sparsity and Regularisation}
\label{subsubsec:Adjusting for Sparsity and Regularisation}


\subsubsection{Algorithm Summary}
\label{subsubsec:Algorithm Summary}



\subsubsection{Adjusting for Bias}
\label{subsubsec:Adjusting for Bias}


\subsection{Item-Based Collaborative Filtering}
\label{subsec:Item-Based Collaborative Filtering}


\subsubsection{Algorithm Formulation}
\label{subsubsec:Algorithm Formulation}

\subsubsection{Algorithm Summary}
\label{subsubsec:Algorithm Summary}


\subsection{User-Based Collaborative Filtering}
\label{subsec:User-Based Collaborative Filtering}


\subsubsection{Algorithm Formulation}
\label{subsubsec:Algorithm Formulation}


\subsubsection{Algorithm Summary}
\label{subsubsec:Algorithm Summary}

\section{Evaluation}
\label{sec:Evaluation}


\subsection{Evaluation Approach}
\label{subsec:Evaluation Approach}


\subsubsection{Predictive Accuracy}
\label{subsubsec:Predictive Accuracy}


\subsubsection{Top-N Evaluation}
\label{subsubsec:Top-N Evaluation}


\subsection{Evaluation Metrics}
\label{subsec:Evaluation Metrics}

\subsubsection{MSE, RMSE and MAE}
\label{subsubsec:MSE, RMSE and MAE}


\subsubsection{HR and NDCG}
\label{subsubsec:HR and NDCG}


\section{Conclusion}
\label{sec:Conclusion for Methodology}
 
 \chapter{Results \& Discussion} % Main chapter title

\label{Chapter5} % Change X to a consecutive number; for referencing this chapter elsewhere, use \ref{ChapterX}

In Chapter \ref{Chapter3} we established the source and details of the Amazon product review dataset and Chapter \ref{Chapter4} addressed the details of the models and techniques adopted in our analysis of building several recommender systems.  This has provided a platform for which we can critically address the primary research questions for this thesis, raised in Section \ref{sec:1 Research Questions and Significance}, by evaluating the various recommender system's predictive accuracy and top-$n$ capability. 

Specifically, this chapter will present the results of the experiments conducted to evaluate the performance of the models described in Chapter \ref{Chapter4}. The results will be presented in the context of the research questions and the previous literature within this domain. We first look at the results of the experiments in Section \ref{sec:5 Results}. Section \ref{sec:5 Discussion} explores the results of the models in the context of the research questions and the previous literature within this domain. The chapter concludes by providing a summary of the results and discussion, in Section \ref{sec:5 Conclusion for Results}.


\section{Results}
\label{sec:5 Results}

In this section we display the results from our analysis, comparing the performance of several different recommenders under two different tasks - rating prediction (Section \ref{subsec:Predictive Accuracy Results}) and top-$n$ evaluation (Section \ref{subsec:5 Top-N Results}). As discussed in Chapter \ref{Chapter4}, the models used in our analysis include the two neighbourhood-based collaborative based approaches (UBCF and IBCF), a model-based collaborative approach (NMF) and the relatively new deep-learning based approach to collaborative filtering (NCF). We built these models for ratings prediction - however, as highlighted in Section \ref{subsec:4 Evaluation Approach}, we evaluated our models under both predictive accuracy and top-$n$ paradigms. 

\subsection{Predictive Accuracy Results}
\label{subsec:Predictive Accuracy Results}

Predictive accuracy, in the context of a recommender system, refers to how well a model is able to predict user preferences or ratings for items \cite{breese2013empirical}. It is usually measured by comparing the predicted ratings to the actual ratings in a dataset. For each user-item pair in the test set, the model makes a prediction about the rating the user would give to the item. The predicted ratings are then compared to the actual ratings in the test set. The measure of how well the model's predictions match the actual ratings is called predictive accuracy \cite{huang2004applying}. The metrics of choice are discussed in detail in Section \ref{sec:2 Evaluation Methods}. Table \ref{tab:pred_acc_res} illustrates the results of our various recommender models. As discussed in Section \ref{sec:4 Modelling Approach}, the NCF model was tested under three different scenarios: using ratings only, using ratings and reviews, and using ratings, reviews, and sentiments. Again, we only include reviews and sentiments in our NCF model - not any of the benchmark models.

\begin{table}[htbp]
    \centering
    \begin{tabular}{|l|l|l|l|l|l|}
    \hline
    \textbf{} & \textbf{Recommender Model} & \textbf{Data Used} & \textbf{MAE} & \textbf{MSE} & \textbf{RMSE} \\
    \hline
    1 &Item-Based Collaborative Filtering & Ratings Only & 0.581 & 0.830 & 0.911 \\
    2 & User-Based Collaborative Filtering & Ratings Only & 0.59 & 0.984 & 0.992 \\
    3& Non-Negative Matrix Factorisation & Ratings Only & 1.583 & 3.467 & 1.862 \\
    4&Neural Collaborative Filtering & Ratings Only & 0.572 & 0.815 & 0.903 \\
    5&Neural Collaborative Filtering & Ratings \& Reviews & \textbf{0.490} & \textbf{0.591} & \textbf{0.769}  \\
    6&Neural Collaborative Filtering & Ratings, Reviews, Sentiments & 0.492 & 0.607 & 0.779 \\
    7&Baseline Model & Ratings Only & 0.67 & 1.270 & 1.311 \\
    \hline
    \end{tabular}
    \caption{Predictive Accuracy Results for Recommender Models Built.}
    \label{tab:pred_acc_res}
\end{table}

The baseline model in Table \ref{tab:pred_acc_res} is a simple model that predicts the average rating for all items. Among these models shown in Table \ref{tab:pred_acc_res}, the NCF model, trained with ratings and reviews (model 5) delivered the most accurate predictions, with an MAE of 0.490, an MSE of 0.591, and an RMSE of 0.769. On the contrary, the NMF model trained with ratings only exhibited the poorest accuracy, recording an MAE of 1.583, an MSE of 3.467, and an RMSE of 1.862. We also noted during our analysis, that both neighbourhood-based approaches, IBCF and UBCF, yielded comparable results. This similarity might be attributed to their shared reliance on the user-item interaction matrix \cite{jia2015user}. 

To determine if the differences in performance between the models are statistically significant, we conducted a paired t-test. This statistical test evaluates whether the differences in the means of two groups are significant or if they could have occurred by random chance \cite{manfei2017differences}. There are a several assumptions that need to be met for the t-test to be valid, including the normality of the data, the homogeneity of variances, the independence of observations and random sampling \cite{manfei2017differences}. For this analysis, we checked whether the assumptions were met and found that they were. Specifically, we found that the data within each group (referring to each recommendation model being evaluated) should follow a normal distribution - the Shapiro-Wilk test\footnote{The Shapiro-Wilk test assesses the normality of data. A p-value greater than the chosen significance level (e.g., 0.05) indicates that the data can be assumed to be normally distributed.} was used to confirm this. We also checked for homogeneity of variances between the groups (referring to the different recommendation models) using Levene's test\footnote{Levene's test evaluates the homogeneity of variances between groups. A non-significant p-value suggests that the variances are approximately equal across groups, which is a prerequisite for t-tests.}. Independence was met since we established consistent data partitioning for each model - so each model was trained and tested on the same data. Lastly, the data was randomly sampled using data partitioning (see Section \ref{sec:3 Data Partitioning}). Thus, we were able to confirm whether the performance differences between NCF with reviews and other models (IBCF, UBCF, and NMF) were statistically significant. The results of the t-test indicated a significant difference in the performance of NCF with reviews (model 5) compared to IBCF ($p$ < 0.01), UBCF ($p$ < 0.01), and NMF ($p$ < 0.01). This confirmed our initial findings that NCF with reviews outperformed other models and that these differences were not due to random chance. 

From Figures \ref{fig:user review distribution} and \ref{fig:product review distribution} in Section \ref{sec:3 Trends and Patterns}, we identified that a key feature of the data is that most products have relatively few reviews (<$50$) and most users have reviewed only a few things (<$20$). This may have implications for the accuracy of the models. For example, the recommender models may be better at predicting ratings for products with more reviews, as it can learn more about the product from the reviews. Similarly, the models may be better at predicting ratings for users with more reviews, as it can learn more about the user from the reviews. To explore this, we can look into the overall accuracy results to see if accuracy depends on any factors - for example, is accuracy higher for users or products with more reviews, or for users with longer review text?

We first look at the impact of the number of reviews per user or item to see if there is a difference in the accuracy of the models for users or items with more reviews. To achieve this, we took users and items in the test set, and assigned them into four groups based on the number of reviews they had. We created these groups by dividing the users and items into quartiles based on the number of reviews they had. We then calculated the average MAE, MSE and RMSE for each group. We can then compare these results to see if the number of reviews has any impact on the accuracy of predictions for users or items who have more reviews. Figures \ref{fig:RMSE_USER_ACC} and \ref{fig:RMSE_ITEM_ACC} show the MSE, RMSE and MAE against the number of reviews for users and items respectively. 

\begin{figure}[htbp]
    \centering
    \begin{subfigure}{0.49\textwidth}
        \centering
        \includegraphics[width=\textwidth]{Figures/ncf_user_ratings_metrics.pdf}
        \caption{Accuracy against the number of reviews for users.}
        \label{fig:RMSE_USER_ACC}
    \end{subfigure}%
    \hfill
    \begin{subfigure}{0.49\textwidth}
        \centering
        \includegraphics[width=\textwidth]{Figures/ncf_item_ratings_metrics.pdf}
        \caption{Accuracy against the number of reviews for items.}
        \label{fig:RMSE_ITEM_ACC}
    \end{subfigure}
    \caption{Comparison of RMSE for user and item ratings.}
    \label{fig:RMSE_ACC}
\end{figure}


From Figures \ref{fig:RMSE_USER_ACC} and \ref{fig:RMSE_ITEM_ACC}, we can see that the for users and items, those with lower review counts appear to have a wide spread in accuracy (for RMSE). The resulting plots for MSE and MAE show similar results. Those users and items which have higher number of reviews, seem to generally achieve better (lower) accuracy scores. This suggests that the recommender model is better at predicting ratings for users with more reviews. This is likely because the models can learn more about the users, since they have more reviews. We also show the mean RMSE, MAE and MSE for each group in Tables \ref{tab:summary_stats_users} and \ref{tab:summary_stats_items} for users and items respectively. For items, we see that the accuracy metrics appear to be similar across the groups. This suggests that the recommender model is not better at predicting ratings for items with more reviews, nor is it worse at predicting ratings for items with fewer reviews.

To test this formally, we conduct an analysis of variance (ANOVA) test to see if the differences in RMSE between the groups are statistically significant. The results of the ANOVA test indicate that the differences in RMSE between the groups are statistically significant for users ($p$ < 0.001). However, they are not statistically significant for items ($p = 0.1$). Thus, to ascertain which groups are significantly different from each other for the users, we conduct a post-hoc test. The post-hoc test we use is the Tukey HSD test\footnote{The Tukey HSD test is a post-hoc test used after an ANOVA test to determine which groups are significantly different from each other. It is used to identify the differences between group means.} which is used to determine which groups are significantly different from each other. The results of the Tukey HSD test are shown in Tables \ref{tab:tukey_hsd_user} in Appendix \ref{AppendixA} for users. We did not use the Tukey HSD test for items as the ANOVA test did not show any statistically significant differences between the groups. 

The results of the Tukey HSD test for users show that the differences in RMSE between the groups are statistically significant, however not for all groups. Specifically, the differences between the low and very high group, and the low and high group are statistically significant. Thus, there is evidence to suggest that the recommender model is better at predicting ratings for users with more reviews. This is likely because the models can learn more about the users, since they have more reviews. This is consistent with previous research that has found that users with more reviews tend to have better accuracy in predicting ratings \cite{srifi2020recommender}. By segmenting the users into groups based on the number of reviews they have, we can see that the accuracy of the models varies across the groups. Effectively, our analysis here suggests that the number of reviews a user has can impact the accuracy of the model's predictions for that user. This finding suggests that the quality of the recommendations accuracy can be influenced by the amount of data available for each user. For items, we did not find any statistically significant differences between the groups. This suggests that the number of reviews an item has does not impact the accuracy of the model's predictions for that item.


\begin{table}[htbp]
    \centering
    \begin{tabular}{lcccc}
        \toprule
        \textbf{Group} & \textbf{Number of Reviews}&\textbf{MAE} & \textbf{MSE} & \textbf{RMSE} \\
        \midrule
        Low & 0-12                &             0.517 &             0.686 &              0.635 \\

        Medium & 13-15               &             0.525 &             0.641  &              0.622 \\

        High & 16-21               &             0.486 &             0.636 &              0.589 \\

        Very High & 21+                 &             0.448 &             0.560 &              0.544 \\

        \bottomrule
    \end{tabular}
    \caption{Mean MAE, MSE, and RMSE for each group of users.}
    \label{tab:summary_stats_users}
\end{table}

\begin{table}[htbp]
    \centering
    \begin{tabular}{lcccc}
        \toprule
        \textbf{Group} & \textbf{Number of Reviews}&\textbf{MAE} & \textbf{MSE} & \textbf{RMSE} \\
        \midrule
        Low & 0-13                &             0.526 &             0.681 &              0.652 \\
        Medium & 14-16               &             0.518 &             0.659 &              0.638 \\

        High & 6-25                &             0.519 &             0.671 &              0.644 \\

        Very High & 26+                 &             0.523 &             0.705 &              0.644 \\

        \bottomrule
    \end{tabular}
    \caption{Mean MAE, MSE, and RMSE for each group of items.}
    \label{tab:summary_stats_items}
\end{table}

Additionally, Figure \ref{fig:distribution of word count} from Section \ref{sec:3 Trends and Patterns} shows that most reviews are relatively short, with a word count of less than 100. We shall explore if this has any implications for the accuracy of the models. For example, is the NCF model better at predicting ratings for reviews with more words? To explore this, we follow a similar approach as previously done. We begin by getting the review length of all the reviews (user-item interactions) that appear in the test set. We then assigned these reviews to four differnt groups, split based on the quartiles. We then calculated the average MAE, MSE and RMSE for each of these groups. Figure \ref{fig:RMSE_reviews} shows the RMSE against the review length in words for users. Table \ref{tab:summary_stats_revs} shows the mean RMSE, MAE and MSE for each group of reviews, assigned based off the review length in words. It appears that reviews with less words have better accuracy in predicting the ratings. Using the ANOVA test, we identify that the differences in RMSE between the groups are statistically significant ($p$<0.001). The resulting Tukey HSD test is shown in Appendix A in Table \ref{tab:tukey_hsd_reviews}. We found that reviews with less words had better accuracy in predicting the ratings. This insight, suggests that the length of the review text can impact the accuracy of the model's predictions. Perhaps, the models are better at predicting ratings for shorter reviews, as they may contain more concise and relevant information. Additionally, reviews with more words may contain more noise or irrelevant information, which could impact the accuracy of the predictions. Effectively, we found that the length of the review text can impact the accuracy of the model's predictions. 

\begin{figure}[htbp]
    \centering
    \includegraphics[width=0.99\textwidth]{Figures/ncf_length_ratings_metrics.pdf}
    \caption{RMSE against the review length in words for users.}
    \label{fig:RMSE_reviews}
\end{figure}


\begin{table}[htbp]
    \centering
    \begin{tabular}{lcccc}
        \toprule
        \textbf{Group} & \textbf{Number of Reviews}&\textbf{MAE} & \textbf{MSE} & \textbf{RMSE} \\
        \midrule
        Low & 0-6                 &             0.262 &             0.276 &              0.262 \\
        Medium & 7-31                &             0.443 &             0.585 &              0.443 \\

        High & 32-118              &             0.610   &             0.808 &              0.601   \\

        Very High & 118+                &             0.682 &             0.878 &              0.682 \\

        \bottomrule
    \end{tabular}
    \caption{Mean MAE, MSE, and RMSE for each group of users based on review length.}
    \label{tab:summary_stats_revs}
\end{table}


\subsection{Top-$N$ Results}
\label{subsec:5 Top-N Results}

As detailed in Section \ref{sec:2 Evaluation Methods}, top-$n$ recommendation represent an important aspect of recommender systems, as they reflect the system's ability to generate relevant suggestions for users \cite{lu2012recommender}. As such, this evaluation paradigm involves evaluating the effectiveness of a recommender system by considering only the top-$n$ recommendations it provides for each user. Our recommender models generated a list of $n$ items for each user. The list of $n$ items were ranked by the system's estimated rating  of each item. We can now evaluate the system's performance based on how many of the items in the top-$n$ list match with items that the user has actually interacted with or rated positively - the test set. The result of which is displayed in Table \ref{tab:top_n_res}. 

\begin{table}[htbp]
    \centering
    \begin{tabular}{|l|l|p{2cm}|l|l|l|}
    \hline
    &\textbf{Model} & \textbf{Data Used} & \textbf{Recall@100} & \textbf{Precision@100} & \textbf{F1-Score@100} \\
    \hline
    1&Item-Based Collaborative Filtering & Ratings Only & 0 & 0.0009 & 0.0001 \\
    2&User-Based Collaborative Filtering & Ratings Only & 0 & 0 & 0 \\
    3&Non-Negative Matrix Factorisation & Ratings Only & 0 & 0.0007 & 0  \\
    4&Neural Collaborative Filtering 1 & Ratings Only & 0.0007 & 0.0010 & 0.0006 \\
    5&Neural Collaborative Filtering 2 & Ratings, Reviews & 0.0007 & 0.0012 & 0.0006 \\
    6&Neural Collaborative Filtering 3 & Ratings, Reviews, Sentiments & 0.0007 & 0.0012 & 0.0006 \\
    \hline
    \end{tabular}
    \caption{Top-N Results for Recommender Models Built.}
    \label{tab:top_n_res}
\end{table}

Recall@100, Precision@100, and F1-Score@100 are metrics used to evaluate the performance of a recommender system in generating a list of top-$n$ (where $n$ is 100) recommendations for each user. As stipulated in \ref{subsec:4 Evaluation Approach}, we execute the top-$n$ evaluation for each user and then aggregate the result, to get a final mean score for each of the metrics. As mentioned, recall@100 represents the proportion of relevant items that were included in the top 100 recommendations, out of all the relevant items in the dataset \cite{cremonesi2010performance}. A higher recall@100 value indicates that the recommender system is doing a better job of capturing relevant items in its recommendations \cite{cremonesi2010performance}. In our case, all the benchmark models  have a recall@100 of 0, meaning that none of the relevant items were included in the top 100 recommendations. Similarly, for NCF models, the majority of users did not have any relevant items in their top-100 recommendations. In contrast, precision@100, which simply represents the proportion of relevant items among the top 100 recommendations \cite{cremonesi2010performance}, has scores generally above 0. The IBCF model has a precision@100 of 0.0009, which means on average, 9 out of every 100 users had 1 of their top 100 recommendations relevant, while the remaining 91 had none. The NMF model has a precision@100 of 0.0007. The NCF models had the highest precision@100 scores - NCF model with reviews (and with sentiments), achieved the highest precision score of 0.0012. Effectively, on average, 12 out of every 100 users had 1 of their top 100 recommendations relevant, while the remaining 88 had none. Finally, the F1-Score@100, which represents the harmonic mean of precision@100 and recall@100, was obtained for each model. Note, a higher F1-score@100 value indicates better overall performance in terms of capturing relevant items in the top recommendations. In our case, IBCF was the only benchmark model to have a value above 0 - F1-score@100 of 0.0001. In contrast, the NCF model's all have an F1-score@100 of 0.0006. 

\section{Discussion}
\label{sec:5 Discussion}

In this section we discuss the practical implications of the study's findings in the context of research questions. We also compare our findings with existing literature and studies in the field, highlighting areas where our results align or diverge from previous research. 

From Table \ref{tab:pred_acc_res} we identified that the family of NCF models outperformed all the other models in terms of predictive accuracy. This suggests that integrating neural networks into the collaborative framework can improve predictive accuracy. The success of NCF models can be attributed to the deep learning techniques they utilise, enabling them to learn complex relationships and patterns in the user-item interaction data \cite{he2017neural} - which they appear to achieve in this problem task.  Neural networks are capable of performing complex non-linear transformations, making them well-suited for handling the high-dimensional nature of the Amazon product review dataset. Additionally, the use of embedding layers in neural networks enables the models to capture latent features and relationships between users and items, leading to more accurate predictions \cite{he2017neural}. 

Regarding the top-$n$ evaluation, we observed that the results were less favourable.  Given the results \ref{tab:top_n_res}, our models do not appear to be doing any better than random\footnote{The probability that a random ranking correctly predicts a relevant product in the top 100 is 1-in-100.} for Top-$n$ prediction. These results can be possibly attributed to two factors. Firstly, our recommender models were primarily designed for ratings prediction rather than top-$n$ recommendations. Effectively, we adapted the models to evaluate top-$n$ recommendations, but they were not optimised for this task. Specifically, our recommender models predict ratings, so for each user we get 11 004 predicted ratings (test set). Now, suppose that only 3 products are relevant (have above the specified threshold rating, 3) for a user, on average. Our adaption of top-$n$ recommendation is then trying to see if the 3 products' ratings will come up in the top 100 of 3249. This is a tough task given that the models were not optimised for. Had the recommender models been redesigned to optimise for top-$n$ recommendations as well, we would expect the results to be very different. Several adjustments would be needed to adapt the current recommender models, such as using a different loss function like Bayesian Personalised Ranking loss, which is designed to optimise for top-$n$ recommendations \cite{rendle2010factorization}.

Secondly, the poor performance of the models in the top-$n$ evaluation may be due to the lack of diversity in the recommendations. The models may be recommending the same popular items to most users, leading to low recall and precision scores. This lack of diversity in recommendations is a common issue in collaborative filtering-based recommenders, as they tend to recommend popular items over more diverse or niche items \cite{adomavicius2005toward}. To address this issue, we can incorporate diversity measures into the recommendation process, such as item diversity or novelty, to ensure that the recommender system is recommending a variety of items to users \cite{steck2013evaluation}. This can help to improve the relevance of the recommendations and increase the chances of capturing relevant items in the top-$n$ recommendations. 

Additionally, it may be a consequence of the dataset's characteristics. From Figure \ref{fig:ratings distribution}, we identified that the ratings distribution was heavily skewed, with most ratings being 5. This skewed distribution may have impacted the system's ability to recommend diverse or niche items. When most ratings are high, it becomes challenging to differentiate between items. The heavily skewed rating distribution, in which most ratings are 5, poses a challenge to the recommender system for top-$n$ recommendations. In such cases, it becomes difficult to differentiate between good items since the predicted ratings are all close to 5. This means that the recommender system may struggle to identify items that stand out from the rest. Consequently, the system may recommend popular, mainstream items over more diverse or niche items that might also be of interest to users. To mitigate the impact of the skewed rating distribution, we can employ several strategies. We could look at incorporating implicit feedback alongside explicit ratings can provide a more comprehensive understanding of user preferences \cite{peska2017using}. Moreover, re-weighting an individual user's rating using their review text could provide a more detailed preference measure of their interest. By adjusting the original ratings, we can achieve a more accurate and reflective feedback that takes into account the nuances expressed in the reviews. This approach leverages the qualitative information provided by users, enabling a more personalised and effective recommendation system. This is another approach for incorporating review text into recommenders that has shown promising results \cite{hariri2011context}.

Ultimately, one of the foremost reasons for the poor top-$n$ results can be attributed to this analysis being focused on predictive accuracy, a task for which the models performed well, rather than for top-$n$ recommendation. Consequently, the poorer results highlighted by Table \ref{tab:top_n_res} underscore the importance of aligning the purpose of the recommender system with its evaluation criteria. This finding resonates with previous research suggesting that there is no universal best recommendation method. Instead, the success of a recommendation system is contingent upon the context and density of available data, with different methods adapting to particular applications being most likely to excel \cite{lu2012recommender}. 

\subsection{Research Questions}
\label{subsec:Research Questions}

We now revisit the research questions posed in Section \ref{sec:1 Research Questions and Significance} and discuss the implications of our findings in the context of these questions.



1. \textbf{How does incorporating product reviews into a recommender system model impact the predictive accuracy and ability to recommend relevant items?} 

The inclusion of review text enhanced the performance of the NCF model, as evidenced by a notable improvement in the MAE, MSE and RMSE for rating prediction. While the improvement in the accuracy of the predictions was significant, it is important to note that the impact on the top-$n$ evaluation was relatively smaller. 

This finding aligns with our hypothesis that leveraging review text alongside ratings data would provide a more comprehensive understanding of user preferences and lead to more accurate recommendations. The results suggest that user reviews contain valuable information that can be utilised to improve the predictive accuracy of recommender systems, especially in tasks such as rating prediction.

2. \textbf{How does incorporating review sentiment as well as review text in a recommender system impact the predictive accuracy and ability to recommend relevant items?}

The results between the NCF model with ratings, review text and review sentiments were comparable to those of the NCF model with review and ratings only. There were no substantial improvements observed with the inclusion of sentiments.

These findings suggest that the sentiment analysis technique used in this study may not have provided significant additional value in the context of the problem being addressed. However, it is important to note that this does not rule out the possibility that a more refined sentiment analysis approach could yield better results in future studies. Future work may involve exploring alternative sentiment analysis techniques or refining the existing one to further investigate the potential impact of incorporating review sentiments into recommender systems. 

In contrast to our results, several studies in the literature have found that incorporating sentiment analysis into recommender systems can improve their performance (\cite{srifi2020recommender}; \cite{roy2022systematic}). This contrast could be due to several factors, such as the specific characteristics of the Amazon product review dataset, as well as the way sentiment analysis was incorporated into the model, or even the model itself - NCF. Moreover, the choice of sentiment analysis was somewhat more sophisticated in the papers mentioned.


3. \textbf{How does the performance of the collaborative-based filtering recommender system using neural collaborative filtering compare to that of popular benchmark recommender systems?}

NCF outperformed all benchmarks for both rating prediction and top-$n$ evaluation metrics. These results signify the potential of neural architecture or deep learning introduction to enhance the accuracy and effectiveness of recommender systems. Research claims that neural networks can capture complex patterns and relationships in data, which is likely the reason for the superior performance of NCF in this study \cite{he2017neural}. Although the results for top-$n$ evaluation were less favourable, hypothesis testing confirmed that the differences in performance between NCF and the benchmark models were statistically significant. 

The superior performance of NCF over other more traditional recommender systems has been widely reported in previous literature (\cite{he2017neural}; \cite{rendle2020neural}). Similarly, other work which have used Amazon product reviews for building recommenders have found that incorporating deep learning-based approached for collaborative filtering have found improved results \cite{rezaei2021amazon}. Specifically, \cite{rezaei2021amazon} found that a deep neural network architecture for collaborative filtering outperformed traditional models in predicting review rating scores. Our findings are consistent with these studies, suggesting that neural collaborative filtering models are more effective in capturing complex user-item interactions and generating more accurate rating predictions.


4. \textbf{What are the potential trade-offs of incorporating product reviews into a recommender system, such as increased complexity or potential biases in the recommendations?}

Having implemented the solution, we acknowledge that incorporating product reviews into a recommender system introduces additional work, but not necessarily increased complexity. These reviews require additional text pre-processing which was discussed in great detail in Section \ref{subsec:3 Text Analysis and Cleaning}, and then finally require text embedding. For NCF, review text is integrated into the input layer alongside user and items.  These embeddings are then concatenated with the user and item embeddings to form the input vector for the neural network - as discussed in Section \ref{sec:4 Modelling Approach}. Within the neural network, the review text is processed alongside the user and item embeddings to generate the final rating prediction. This process is not necessarily more complex than the existing models.

To explore the computational runtimes associated with incorporating reviews compared to those models that were built without reviews, we monitored the training run times. Table \ref{tab:computational-details} provides a comparison of the run-times of various collaborative filtering models. The benchmark models show higher runtimes compared to all the NCF model instances, indicating that they may require more computational resources compared to the NCF model. This is likely due to the fact that these benchmark models were built from scratch using Python, while the NCF model was built using TensorFlow, which is optimised for deep learning tasks. Additionally, incorporating the reviews into the recommender model did increase the run time of the NCF model from 12 minutes to 13 minutes. Moreover, we note that including sentiments as well as reviews to the NCF model did not increase the computation run time compared to using reviews and ratings only. 

Overall, incorporating product reviews into a recommender system, appears to not greatly impact the runtime of the model. This is likely due to the additional complexity of incorporating reviews is not significantly higher than the existing models. Furthermore, NCF models in general, even with the additional augmented review and sentiments data run faster (13 minutes) than the quickest benchmark model - UBCF at 125 minutes. Again, this is largely due to the fact that the NCF model is built using TensorFlow, which is optimised for deep learning tasks, whilst the benchmark models were built from scratch using Python. 

\begin{table}[htbp]
    \centering
    \begin{tabular}{|l|l|l|}
    \hline
    \textbf{Model} & \textbf{Data Used} & \textbf{Runtime (minutes)} \\
    \hline
    Item-Based Collaborative Filtering & Ratings Only & 140 \\
    User-Based Collaborative Filtering & Ratings Only & 125 \\
    Non-Negative Matrix Factorisation & Ratings Only & 457 \\
    Neural Collaborative Filtering 1 & Ratings Only & 12 \\
    Neural Collaborative Filtering 2 & Ratings \& Reviews &13 \\
    Neural Collaborative Filtering 3 & Ratings, Reviews, Sentiments & 13 \\
    \hline
    \end{tabular}
    \caption{Computational Details of Various Collaborative Filtering Models}
    \label{tab:computational-details}
\end{table}

\section{Conclusion}
\label{sec:5 Conclusion for Results}

In this chapter, we presented the outcomes of our various recommender models and their implications within the context of our research questions, contrasting them with existing literature on review-aware recommenders. Effectively, this chapter is the culmination of the literature review (Chapter \ref{Chapter2}), data exploration (Chapter \ref{Chapter3}), and methodological approach (Chapter \ref{Chapter4}).

Most notably, the NCF model outperformed all the benchmark models in terms of both predictive accuracy and top-$n$ evaluation. Our findings closely align with existing research on incorporating deep learning-based techniques into the collaborative filtering network \cite{he2017neural}, where there has been some significant attention and improvements discovered. Additionally, one-sampled paired t-tests revealed that all improvements were statistically significant at a significance level of $p < 0.01$ for predictive accuracy results. Moreover, we note that the review-aware NCF model augmented with sentiments did not exhibit significant performance improvements in terms of predictive accuracy. Additionally, top-$n$ evaluation results were discussed in detail, focusing on addressing the less favourable results observed across all the recommenders. We emphasised the importance of tailoring a recommender for a specific task. Since our primary concern was predictive accuracy, our recommenders yielded  poor results in top-$n$ evaluation - often not performing better than random. Strategies for improvement were outlined, and additional factors contributing to the poor results were also addressed - namely the skew rating distribution. Effectively, this chapter highlights the importance of incorporating reviews into recommender systems, particularly through the review-aware NCF model, which provided evidence that incorporating reviews improves predictive accuracy. Although there was no substantial improvement with the inclusion of sentiments, the overall performance of the NCF model suggests promising opportunities for future exploration and optimisation. 

Importantly, we have addressed each of the four research questions posed in Section \ref{sec:1 Research Problem and Objectives}. To summarise, incorporating reviews into a recommender model does lead to an improvement in predictive accuracy. Incorporating sentiments, did not yield improved results, however there is ample opportunities for further exploration, especially in choosing more sophisticated sentiment extraction techniques. The NCF model consistently outperformed all the benchmark models, showing improvements in both predictive accuracy and top-$n$ evaluation, suggesting that the deep learning-based approach is more effective in capturing complex user-item interactions than the traditional collaborative filtering methods. Finally, incorporating reviews into a recommender system did not yield large increases in the computational complexity of the models, evidenced by the marginal increase in run times for the NCF model. 
 
  % Chapter Template

\chapter{Conclusion} % Main chapter title
\label{Chapter6} % Change X to a consecutive number; for referencing this chapter elsewhere, use \ref{ChapterX}

In Chapter \ref{Chapter5}, we looked into the results from this thesis and their implications concerning our research questions, including a deeper discussion of what was observed from our analysis in general. In this chapter, we aim to offer a comprehensive overview of the primary findings derived from this thesis, representing the culmination of our analysis, and outlining their implications moving forward.

The chapter begins with an overview of the results in Section \ref{sec:6 Summary of Results and Findings}. Following this, Section \ref{sec:6 Limitations} addresses the limitations of the thesis, both from a perspective of limited experimentation, as well as from a perspective of the various assumptions made. These identified shortcomings pave the way for Section \ref{sec:6 Future Work}, which explores potential avenues for further exploration or enhancement of the methodologies employed. Finally, we draw this thesis to a close with Section \ref{sec:6 Conclusion for Conclusion}, which serves to summarise and raise key points to conclude this thesis. 


\section{Summary of Results and Findings}
\label{sec:6 Summary of Results and Findings}


In Section \ref{sec:5 Results}, we evaluated the results from our study for two problems: rating prediction and top-$n$ generation. We summarise the key findings from our study. 



\textbf{Rating Prediction}
\begin{itemize}
    \item The NCF model outperformed all benchmark models across all predictive accuracy metrics, including MAE, RMSE, and MSE.
    \item The review-aware NCF model, integrating ratings and review text, exhibited the best performance, achieving an MAE of 0.490 and RMSE of 0.769.
    \item The inclusion of sentiment analysis alongside reviews marginally decreased performance, resulting in marginally higher RMSE and MAE values (MAE of 0.492 and RMSE of 0.779).
    \item A one-sample paired t-tests showed that there is evidence that the NCF model significantly outperforms all benchmark models at a significance level of $p<0.01$.
    \item Users with more reviews had lower MAE and RMSE values, indicating that the NCF model performed better for users with more reviews.
    \item The predictive accuracy did not change significantly for items with more reviews, indicating that the NCF model performed similarly for items with varying review counts.
    \item  Reviews with fewer words had lower MAE and RMSE values, indicating that the NCF model performed better for reviews with fewer words.
    \end{itemize}
    

\textbf{Top-$N$ Generation}
\begin{itemize}
    \item The NCF model struggled to recommend relevant products for users' top-$n$ recommendations, particularly within the Top-100 list.
    \item All models performed poorly on the top-$n$ tasks, although the NCF model outperformed the benchmarks models.
\end{itemize}


\textbf{Computational Details}
\begin{itemize}
    \item The NCF model with review text and sentiments required only 13 minutes for model training
    \item The slowest NCF model, NCF model with reviews, was quicker (13 minutes) than the fastest benchmark model, user-based collaborative filtering (125 minutes).
    \item NMF took the longest runtime at 457 minutes.
    \item The efficiency of the NCF models was particularly noteworthy with respect to computation, with shorter run-times compared to benchmark models, even when augmented with additional information such as review text and sentiments. Largely attributed to using TensorFlow for NCF, whilst the benchmark models were built from scratch.
\end{itemize}


Directly addressing our research questions, the incorporation of review text enhanced rating prediction performance within recommender systems. However, the addition of sentiments did not yield further improvements. Moreover, the efficacy of the NCF models was evident, performing better than all the benchmark models across the predictive accuracy metrics used. Furthermore, the efficiency of the NCF models was particularly noteworthy with respect to computation, with shorter run-times compared to benchmark models, even when augmented with additional information such as review text and sentiments. 


\section{Limitations}
\label{sec:6 Limitations}

In building our recommender system, our primary focus centered on rating predictions aimed at minimising the disparity between actual and predicted ratings. The overarching goal was to develop a recommendation system capable of accurately predicting unknown ratings for items, leveraging a training set to discern user preferences, and assessing the system's efficacy by evaluating its performance on a testing set containing concealed ratings provided by users for various items. This evaluation process involved measuring predictive accuracy metrics such as RMSE and MAE. Having achieved predicted ratings for all unrated items for each user we were able to identify a list of items (top-$n$) by sorting the ratings for unrated items to identify a top-$n$ list of recommendations for each user. We measured this list using metrics such as recall@$n$ and precision@$n$. The results of which were not impressive. In hindsight, these observations shed light on several limitations inherent in our study, prompting a critical re-evaluation of our methodologies and approaches.

Firstly, we built the model for a specific goal - rating prediction, however we tested its capability on different task - top-$n$ generation. We have established that there is no one universal recommender system \cite{lu2012recommender} and that a recommender's performance is not transferable from one objective to another. Consequently, the results obtained from the top-$n$ generation task offer limited insights into the effectiveness of our review-aware NCF model in this specific use case. Indeed, while our recommender system performed relatively well in predicting ratings for unknown products, its performance in generating top-$n$ recommendations was considerably less satisfactory. That is to say, we observed that the model struggled to recommend relevant (items in test set) products for a users Top-100. In fact, the performance was not better than random recommender. Thus, given the misalignment between the model's training objective and the evaluation task, the results obtained from the top-$n$ generation task should be interpreted with caution, and the conclusions drawn from them should be viewed in light of this limitation. 

To that end, the misalignment between the training objective and the evaluation task underscores the importance of aligning the recommender system's training and evaluation objectives to ensure that the model is optimised for the task at hand. With the evolving understanding that recommendation accuracy alone does not guarantee an effective and satisfying user experience, our approach of generating a recommender solely for predicting unrated items provides a limited scope for building a good or useful recommender system. To address this limitation, it becomes imperative to extend beyond simple accuracy metrics and optimise recommenders for tasks beyond rating prediction. This necessitates a more nuanced evaluation approach that encompasses multiple dimensions of performance, rather than focusing solely on rating prediction accuracy. Such multi-objective recommenders can be designed to optimise for a range of criteria, including diversity, novelty, serendipity, and user satisfaction, thereby ensuring a more comprehensive recommendation capability. 


Another key challenge and limitation encountered in this study pertained to the computational resources allocated to meet the thesis's requirements. For loading and handling the dataset, we relied on our local machine, which presented several challenges in managing the data effectively and efficiently. Thus, we limited our dataset by sampling records from the original repository, which contained over 140 million records. Additionally, we further restricted the dataset to users with 13 or more ratings and items with 13 or more reviews. While these constraints were necessary to ensure the feasibility of our analysis on the local machine, they also introduced limitations that may have impacted the performance and utility of our recommender system. 

With respect to the computational restrictions, beyond dataset handling, the model building and training occurred on our local machine, which inhibited the path of additional experimentation. Despite our efforts, all the benchmark models were restricted (due to their excessive run-times and memory allocations) in their tuning capabilities, with hyperparameter adjustments limited to only two or three options for each parameter. While it is not expected that the performance will be substantially improved had we been able to perform a more extensive hyperparameter search, it could be worthwhile trying to quantify how much improvement can be gained by performing a large hyperparameter search as compared to using a standard set of hyperparameters. Additionally, we compared an NCF model with extensive hyperparameter tuning with benchmark models that had minimal hyperparameter tuning. This comparison could have been more equitable had we performed a more extensive hyperparameter search for the benchmark models. Ultimately, the computational constraints as well as the limited time dictated a lot of the decisions made in the methodology. This challenge also touches upon a broader issue of scalability, an aspect that was acknowledged in Section \ref{subsec:2 Scalability} but was not addressed within the scope of this thesis. Nonetheless, given the context of the recommender system within e-commerce, scalability emerges as a primary concern. Addressing this concern could significantly enhance the run-times of benchmark models, particularly matrix factorisation approaches, by leveraging packages specifically designed to handle the inherent sparsity of the user-item matrix. The benchmark models, detailed in Section \ref{sec:4 Benchmark Models}, were constructed from scratch in Python, with minimal effort directed towards mitigating scalability issues or managing the computational overhead of training. 

One final limitation inherent in our thesis is the omission of addressing the cold start problem. We chose to mitigate the cold start problem, or rather circumvent it entirely, by restricting the dataset to users with 13 or more ratings and items with 13 or more reviews. While this approach provides a starting point, it limits the generalisability of our findings and the applicability of our recommender system to real-world scenarios. The cold start problem is a pervasive challenge in recommender systems, particularly for new users or items with limited interaction history. Addressing this challenge requires the development of innovative strategies to handle users or items with limited historical data. While the cold start problem was not the primary focus of our study, it represents a critical limitation that warrants further exploration and consideration in future research.


\section{Future Work}
\label{sec:6 Future Work}

\subsection{Developing Multi-Objective Recommender Systems}
\label{subsec:6 Multi-Objective Recommender Systems}

We have stablished the importance of a comprehensive evaluation approach to recommender systems, one that extends beyond rating prediction accuracy to encompass a broader range of performance metrics. Avenues for developing multi-objective recommenders that optimise for diverse criteria, including diversity, novelty, serendipity, and user satisfaction, represent a promising direction for future research. By adopting a more holistic evaluation approach, recommender systems can be designed to cater to a wider range of user preferences and needs, thereby enhancing the overall user experience. This approach aligns with the evolving understanding that recommendation accuracy alone does not guarantee an effective and satisfying user experience, underscoring the need for a more nuanced evaluation framework that captures the multifaceted nature of recommender systems. Various strategies can be employed to address this, including direct enhancement of recommendation list diversity and the integration of hybrid recommendation methods to meet different task objectives (\cite{smyth2001similarity};\cite{ziegler2005improving};\cite{hurley2011novelty};\cite{zhou2010solving}). Ultimately, future endeavors should prioritise the adoption of a more comprehensive approach to developing multi-objective recommender systems,and hence, evaluating recommenders based on a broader range of performance metrics.

\subsection{Enhancing Neural Collaborative Filtering with NeuMF}
\label{subsec:6 Enhancing Neural Collaborative Filtering with NeuMF}

The development of our NCF was based off the framework established by \cite{he2017neural}. The findings from our analysis underscored the improvements in predictive accuracy achieved by the NCF model when compared to conventional collaborative filtering methods. The work in this thesis can be extended to hybridising the architecture of NCF by incorporating matrix factorisation, resulting in the algorithm Neural Matrix Factorisation (NeuMF) - as detailed by \cite{he2017neural}. The rationale behind this approach stems from the recognition that traditional matrix factorisation can be viewed as a specialised instance of NCF. Therefore, by fusing the neural interpretation of matrix factorisation with Multilayer Perceptron (MLP), NeuMF emerges as a more generalised model harnessing the linearity of matrix factorisation and the non-linearity of MLP to enhance recommendation quality. Notably, the empirical findings from studies such as \cite{zhang2019deep} and \cite{he2017neural} corroborate the performance benefits offered by NeuMF over NCF. This naturally extends our comparative analysis, inviting us to integrate and adapt the advancements put forth by \cite{he2017neural} to incorporate NeuMF into our framework. This avenue for future can further be extended by investigating the augmentation of review text and sentiments within this enhanced framework — an area that, to the best of our knowledge, has not yet been explored.


\subsection{Exploring Advanced Word Embedding and Sentiment Analysis Techniques}

For our analysis, we employed a simple word embedding technique (USE) to convert review text into numerical vectors, which were subsequently integrated into our NCF model. However, the choice of word embedding technique can significantly impact the performance of the recommender system \cite{asudani2023impact}, with advanced methods such as Word2Vec, GloVe, and BERT offering more sophisticated representations of textual data (\cite{mikolov2013distributed}; \cite{pennington2014glove}; \cite{devlin2018bert}). Therefore, it stands to reason that exploring additional word embedding techniques beyond the simple implementation undertaken in this thesis would represent a natural extension of our research efforts. Similarly, the sentiment analysis component of our study was based on a lexicon-based approaches only, which may not capture the full complexity and nuances of user sentiments. By incorporating more advanced sentiment analysis techniques, such as deep learning-based models, we can possibly enhance the accuracy and granularity of sentiment analysis within our recommender system. Notably, the choice of sentiment analysis technique has been documented as an important determinant in feature creation for subsequent analysis in existing literature \cite{ahuja2019impact}. 

Ultimately, the integration of advanced word embedding and more sophisticated sentiment analysis techniques can enrich the insights derived from user reviews, thereby enhancing the effectiveness of the recommender system. Such an approach can perhaps lead to more representative sentiments from the review text, thereby furnishing the model with richer insights into user preferences. Thus, by experimenting with more methodologies, there is an opportunity to further enhance the effectiveness of our recommender system.



\subsection{Addressing the Cold Start Problem}
\label{subsec:6 Addressing the Cold Start Problem}

A natural extension to our analysis is to address the cold start problem, a challenge in recommender systems that arises when new users or items with limited interaction history are introduced. While we circumvented this issue by restricting our dataset to users with 13 or more ratings and items with 13 or more reviews, this approach limits the generalisability of our findings. To address the cold start problem, innovative strategies can be employed to handle users or items with limited historical data, thereby enhancing the robustness and versatility of the recommender system. For instance, hybrid recommendation approaches that combine collaborative filtering with content-based filtering can be employed to mitigate the cold start problem by leveraging the strengths of both methods. Content-based filtering methods can be leveraged to provide recommendations based on item attributes or user profiles, thereby circumventing the need for historical interaction data \cite{claypool1999combing}. By addressing the cold start problem, the developed recommender systems can cater to a wider range of users and items, thereby enhancing their utility and effectiveness in real-world scenarios. This avenue for future research represents a critical step towards developing more comprehensive and generalisable recommender systems capable of handling the challenges posed by the cold start problem.

\subsection{Leveraging Additional Data Modalities}
\label{subsec:6 Leveraging Additional Data Modalities}

The Amazon Product Review Data (Section \ref{subsec:3 Amazon Review Dataset}) utilised in this study presents additional opportunities for experimentation. Notably, each product within the dataset is accompanied by an image feature — an input that has garnered attention in research within several industries, particularly fashion E-commerce, where images have been leveraged to enrich the top-$n$ generation process (\cite{tuinhof2019image};\cite{kurt2017image}). By augmenting our recommender system to incorporate an additional data modality, such as images, there is opportunity for enhancing the efficacy of the model, building upon the foundation laid in this thesis. Furthermore, the dataset offers a valuable temporal dimension, with each review provided by a user being timestamped, spanning a considerable time frame from 1996 to 2018. Harnessing this temporal aspect and tracking trends in user preferences over time could furnish the recommender system with valuable insights, enabling it to adapt and evolve in response to shifting user preferences. This not only facilitates a more comprehensive evaluation of the recommender system but also ensures its relevance and efficacy under realistic and dynamic conditions. 

\section{Conclusion}
\label{sec:6 Conclusion for Conclusion}


Recommender systems play a crucial role in modern digital landscapes, acting as indispensable tools that guide users towards items they are likely to find valuable based on their individual preferences and the vast array of available options \cite{jannach2010recommender}. By doing so, these systems empower users to efficiently navigate through many options, connecting them with products, services, or knowledge that resonate with their needs and interests \cite{jannach2010recommender}. This ability to streamline the decision-making process has large implications for businesses, ranging from increased cross-selling opportunities to heightened customer satisfaction, ultimately translating into tangible gains in revenue and market competitiveness \cite{leino2007case}.

Collaborative filtering stands as one of the cornerstone paradigms within the realm of recommender systems \cite{burke2015robust}. At its core, collaborative filtering operates on the principle of leveraging user-generated ratings and feedback to generate personalised recommendations. By harnessing the collective 'wisdom' (history) of users who have rated items they've interacted with, this technique effectively identifies patterns and similarities among user preferences \cite{burke2015robust}. This approach, traditionally involved calculating and using similarity metrics, or perhaps matrix factorisation methods to build models to predict interest. However, there has been a growing interest in adapting traditional collaborative filtering framework to incorporate deep learning-based approaches - one such approach is neural collaborative filtering \cite{he2017neural}.


The focal point of this thesis was centered on harnessing a neural network-based approach for collaborative filtering, NCF, with the objective of adapting the framework provided by \cite{he2017neural} to accommodate nuanced user preferences by incorporating review text and sentiments. The primary goals of the thesis were to assess whether the incorporation of a neural network architecture in collaborative filtering enhances performance and whether augmenting a recommender with review text and sentiments yields similar improvements. Leveraging the Amazon product review dataset, extensive text cleaning and pre-processing, including word embedding and sentiment analysis, were conducted to prepare the data for integration into our neural architectures within the recommender models. Adhering to the widely adopted leave-$k$-out approach, we partitioned our data and designated 3 items from each user to form the test set. Our evaluation involved comparing the performance of our NCF systems with that of traditional collaborative filtering methods, namely IBCF, UBCF, as well as NMF. All recommender models were constructed and trained for rating prediction task, with the NCF model featuring a relatively simple MLP neural architecture augmented with additional layers to accommodate additional data inputs. Evaluation metrics such as MAE and RMSE were employed to assess the models' rating prediction capabilities. Additionally, we leveraged predicted ratings to generate top-$n$ recommendations for each user and evaluated the recommender systems' effectiveness using recall@$n$ and precision@$n$ metrics.

The outcomes of this study underscored that incorporating review text into the NCF model enhances its predictive accuracy, with the NCF model with reviews and ratings outperforming all benchmark models across all predictive accuracy metrics. Incorporating sentiments alongside review text did not yield further improvements, with the sentiment-augmented NCF model exhibiting marginally higher RMSE and MAE values. For the top-$n$ generation task, all the models struggled to recommend relevant products for users within the Top-100 list, however, the NCF models outperformed the benchmark models for this task. Additionally, we found that incorporating sentiments alongside review text did not increase the runtimes of the models significantly. 

This thesis concluded by addressing both the limitations inherent in the study and outlined potential avenues for future research that could build upon the current efforts. One of the primary limitations identified was the misalignment between the training objective and the evaluation task, which underscored the importance of aligning the recommender system's training and evaluation objectives to ensure that the model is optimised for the task at hand. To that end, potential avenues for future research include the development of multi-objective recommender systems. In addition, we addressed possibly enhancing the NCF model with NeuMF, exploring advanced word embedding and sentiment analysis techniques, addressing the cold start problem, and leveraging additional data modalities. These avenues represent promising directions for future research that could build upon the current study's findings and enhance the effectiveness and utility of recommender systems in real-world scenarios.

In culmination, this thesis aimed to make incremental contributions to the evolving landscape of recommender systems, with a specific focus on harnessing textual information to augment recommendation accuracy. While our endeavor sought to pave the way for enhanced recommendation accuracy, we acknowledge the hurdles and constraints inherent in crafting recommendation systems capable of adeptly catering to user preferences and item characteristics. The avenues for future work lay bare and offer great potential for expanding the scope of our endeavors and enriching recommendation algorithms. 
  % Chapter Template

\chapter{Discussion} % Main chapter title

\label{Chapter7} % Change X to a consecutive number; for referencing this chapter elsewhere, use \ref{ChapterX}


%----------------------------------------------------------------------------------------
%	SECTION 1
%----------------------------------------------------------------------------------------

\section{Example}



%----------------------------------------------------------------------------------------
%	SECTION 2
%----------------------------------------------------------------------------------------

\section{Example}




%----------------------------------------------------------------------------------------
%	SECTION 3
%----------------------------------------------------------------------------------------

\section{Example}


 




%----------------------------------------------------------------------------------------
%	THESIS CONTENT - APPENDICES
%----------------------------------------------------------------------------------------

\appendix 

% Appendix A

\chapter{} % Main appendix title

\section{Example}


% Appendix A

\chapter{} % Main appendix title
\section{Example}

\newpage
\section{Example}
% Appendix A



\label{} % For referencing this appendix elsewhere, use \ref{AppendixA}

\section{Random Forest Algorithm}

Random Forest for Regression or Classification (Brillinger, 2001).

\begin{enumerate}
\item For $b=1$ to $B$
	\item Draw a bootstrap sample $\mathbf{Z}^{*}$ of size $N$ from the training data.
	\item Grow a random-forest tree $T_{b}$ to the bootstrapped data, by re- cursively repeating the following steps for each terminal node of the tree, until the minimum node size $n_{\min }$ is reached.
		\item Select $m$ variables at random from the $p$ variables.
		\item Pick the best variable/split-point among the $m$.
		\item Split the node into two daughter nodes.

\item Output the ensemble of trees $\left\{T_{b}\right\}_{1}^{B}$.
\end{enumerate}

To make a prediction at a new point $x$ :
\begin{itemize}
\item Regression: $\hat{f}_{\mathrm{rf}}^{B}(x)=\frac{1}{B} \sum_{b=1}^{B} T_{b}(x)$.\\
\item Classification: Let $\hat{C}_{b}(x)$ be the class prediction of the $b$ th random tree. Then $\hat{C}_{\mathrm{rf}}^{B}(x)=$ majority vote $\left\{\hat{C}_{b}(x)\right\}_{1}^{B}$.
\end{itemize}

\section{Technical Indicators Used in Analysis}

The following technical indicators were used in. our study. The definitions were scraped from Lo and Hasanhodzic (2010).

\textbf{Standard Deviation} (SD): measures market volatility and is used in statistics to describe the variability or dispersion of a set of data around the average.



\textbf{Relative Strength Index} (RSI): is a momentum indicator used in technical analysis that measures the magnitude of recent price changes to evaluate overbought or oversold conditions in the price of a stock or other asset.


\textbf{Average Directional Index }(ADX): is a technical analysis indicator used by some traders to determine the strength of a trend.


\textbf{Moving Average}(MA): is a stock indicator that is commonly used in technical analysis. The reason for calculating the moving average of a stock is to help smooth out the price data over a specified period of time by creating a constantly updated average price.



\textbf{Exponential Moving Average }(EMA): is a technical chart indicator that tracks the price of an investment (like a stock or commodity) over time.

\textbf{Momentum} (MTM): is a technical indicator which shows the trend direction and measures the pace of the price fluctuation by comparing current and past values.

\textbf{Average True Range} (ATR): is a market volatility indicator used in technical analysis. It is typically derived from the 14-day simple moving average of a series of true range indicators.


\textbf{Bollinger Bands} (BB): are envelopes plotted at a standard deviation level above and below a simple moving average of the price.


\textbf{Stochastic Oscillator} (SO): is a momentum indicator comparing a particular closing price of a security to a range of its prices over a certain period of time.


\textbf{Triple Exponential Average} (TRIX): is an oscillator used to identify oversold and overbought markets, and it can also be used as a momentum indicator.


\textbf{Moving Average Convergence Divergence} (MACD): It is designed to reveal changes in the strength, direction, momentum, and duration of a trend in a stock's price. 



\textbf{Mass Index }(MI): used in technical analysis to predict trend reversals.


\textbf{Vortex Indicator }(VI): is used to spot trend reversals and confirm current trends.


\textbf{True Strength Index} (TSI): is a technical momentum oscillator used to identify trends and reversals.

\textbf{Chaikan Oscillator} (CO): is the difference between the 3-day and 10-day EMAs of the Accumulation Distribution Line.


\textbf{Force Index }(FI): measures the amount of power used to move the price of an asset.

\textbf{Ease of Movement} (EOM): is a technical study that attempts to quantify a mix of momentum and volume information into one value.


\textbf{Commodity Channel Index }(CCI): is a technical indicator that measures the difference between the current price and the historical average price.

\textbf{Keltner Channel }(KC): technical indicator that day traders can use to help assess the current trend and provide trading signals.

\textbf{Ultimate Oscillator} (UO): is a range-bound indicator with a value that fluctuates between 0 and 100.  Used to identify oversold or over bought assets.



% Appendix A

\chapter{Boosting Grid Search Learning Rates} % Main appendix title







\begin{figure}[h]
    \centering
    \subfloat[Learning Rate: 0.01]{\label{fig:2:a}\includegraphics[width=0.6\textwidth]{/Users/pavansingh/Google Drive (UCT)/STA Honours/Project/Thesis/Python Coding/Figures/RF AND BOOST/VAL/Best_Model_Boosting_0.pdf}}\\
    \subfloat[Learning Rate: 0.1]{\label{fig:2:b}\includegraphics[width=0.6\textwidth]{/Users/pavansingh/Google Drive (UCT)/STA Honours/Project/Thesis/Python Coding/Figures/RF AND BOOST/VAL/Best_Model_Boosting_1.pdf}}    \\
    \subfloat[Learning Rate: 1]{\label{fig:2:d}\includegraphics[width=0.6\textwidth]{/Users/pavansingh/Google Drive (UCT)/STA Honours/Project/Thesis/Python Coding/Figures/RF AND BOOST/VAL/Best_Model_Boosting_2.pdf}} \\
    \caption{\small{Grid search results for using different learning rates on our AdaBoost model.}}
\label{}
\end{figure}


\chapter{Validation and Loss Curves for LSTM} % Main appendix title


\begin{figure}[h]
    \centering
    \subfloat[Multivariate LSTM Ford]{\label{fig:2:a}\includegraphics[width=0.6\textwidth]{/Users/pavansingh/Google Drive (UCT)/STA Honours/Project/Thesis/Python Coding/Figures/ANN/LOSS/ModelLoss_ANN_Ford.pdf}}\\
    \subfloat[Multivariate LSTM Ford]{\label{fig:2:b}\includegraphics[width=0.6\textwidth]{/Users/pavansingh/Google Drive (UCT)/STA Honours/Project/Thesis/Python Coding/Figures/ANN/LOSS/ModelLoss_ANN_Motorola.pdf}}    \\
    \subfloat[Multivariate LSTM Ford]{\label{fig:2:d}\includegraphics[width=0.6\textwidth]{/Users/pavansingh/Google Drive (UCT)/STA Honours/Project/Thesis/Python Coding/Figures/ANN/LOSS/ModelLoss_ANN_Google.pdf}} \\
    \caption{\small{Grid search results for using different learning rates on our AdaBoost model.}}
\label{}
\end{figure}
% Appendix A

\chapter{Confusion Matrices} % Main appendix title




\begin{figure}[h]
    \centering
    \subfloat[Random Forest]{\label{fig:2:a}\includegraphics[width=0.3\textwidth]{/Users/pavansingh/Google Drive (UCT)/STA Honours/Project/Thesis/Python Coding/Figures/RF AND BOOST/CONF/ConfidenceMatrix_RF_Google.pdf}}
    \subfloat[AdaBoost]{\label{fig:2:b}\includegraphics[width=0.3\textwidth]{/Users/pavansingh/Google Drive (UCT)/STA Honours/Project/Thesis/Python Coding/Figures/RF AND BOOST/CONF/ConfidenceMatrix_Ada_Google.pdf}}    
    \subfloat[Univariate Feed-Forward]{\label{fig:2:d}\includegraphics[width=0.3\textwidth]{/Users/pavansingh/Google Drive (UCT)/STA Honours/Project/Thesis/Python Coding/Figures/ANN/CONF/ConfidenceMatrix_FFNN_Uni_Google.pdf}} \\
    \subfloat[Multivariate Feed-Forward]{\label{fig:2:e}\includegraphics[width=0.3\textwidth]{/Users/pavansingh/Google Drive (UCT)/STA Honours/Project/Thesis/Python Coding/Figures/ANN/CONF/ConfidenceMatrix_FFNN_Multi_Google.pdf}}    
    \subfloat[Univariate LSTM]{\label{fig:2:f}\includegraphics[width=0.3\textwidth]{/Users/pavansingh/Google Drive (UCT)/STA Honours/Project/Thesis/Python Coding/Figures/LSTM/CONF/ConfidenceMatrix_LSTM_Uni_Google.pdf}}
     \subfloat[Multivariate LSTM]{\label{fig:2:f}\includegraphics[width=0.3\textwidth]{/Users/pavansingh/Google Drive (UCT)/STA Honours/Project/Thesis/Python Coding/Figures/LSTM/CONF/ConfidenceMatrix_LSTM_Multi_Google.pdf}}
    \caption{\small{Confusion matrixes for the various machine learning models for Google only. Left axes depicts the true movement of the daily log returns, while the bottom axes depicts the predicted movement}}
\label{}
\end{figure}



\begin{figure}[h]
    \centering
    \subfloat[Random Forest]{\label{fig:2:a}\includegraphics[width=0.3\textwidth]{/Users/pavansingh/Google Drive (UCT)/STA Honours/Project/Thesis/Python Coding/Figures/RF AND BOOST/CONF/ConfidenceMatrix_RF_Motorola.pdf}}
    \subfloat[AdaBoost]{\label{fig:2:b}\includegraphics[width=0.3\textwidth]{/Users/pavansingh/Google Drive (UCT)/STA Honours/Project/Thesis/Python Coding/Figures/RF AND BOOST/CONF/ConfidenceMatrix_Ada_Motorola.pdf}}    
    \subfloat[Univariate Feed-Forward]{\label{fig:2:d}\includegraphics[width=0.3\textwidth]{/Users/pavansingh/Google Drive (UCT)/STA Honours/Project/Thesis/Python Coding/Figures/ANN/CONF/ConfidenceMatrix_FFNN_Uni_Motorola.pdf}} \\
    \subfloat[Multivariate Feed-Forward]{\label{fig:2:e}\includegraphics[width=0.3\textwidth]{/Users/pavansingh/Google Drive (UCT)/STA Honours/Project/Thesis/Python Coding/Figures/ANN/CONF/ConfidenceMatrix_FFNN_Multi_Motorola.pdf}}    
    \subfloat[Univariate LSTM]{\label{fig:2:f}\includegraphics[width=0.3\textwidth]{/Users/pavansingh/Google Drive (UCT)/STA Honours/Project/Thesis/Python Coding/Figures/LSTM/CONF/ConfidenceMatrix_LSTM_Uni_Motorola.pdf}}
     \subfloat[Multivariate LSTM]{\label{fig:2:f}\includegraphics[width=0.3\textwidth]{/Users/pavansingh/Google Drive (UCT)/STA Honours/Project/Thesis/Python Coding/Figures/LSTM/CONF/ConfidenceMatrix_LSTM_Multi_Motorola.pdf}}
    \caption{\small{Confusion matrixes for the various machine learning models for Motorola only. Left axes depicts the true movement of the daily log returns, while the bottom axes depicts the predicted movement}}
\label{}
\end{figure}





% % Appendix A

\chapter{} % Main appendix title
\section{Example}

\newpage
\section{Example}
% % Appendix A



\label{} % For referencing this appendix elsewhere, use \ref{AppendixA}

\section{Random Forest Algorithm}

Random Forest for Regression or Classification (Brillinger, 2001).

\begin{enumerate}
\item For $b=1$ to $B$
	\item Draw a bootstrap sample $\mathbf{Z}^{*}$ of size $N$ from the training data.
	\item Grow a random-forest tree $T_{b}$ to the bootstrapped data, by re- cursively repeating the following steps for each terminal node of the tree, until the minimum node size $n_{\min }$ is reached.
		\item Select $m$ variables at random from the $p$ variables.
		\item Pick the best variable/split-point among the $m$.
		\item Split the node into two daughter nodes.

\item Output the ensemble of trees $\left\{T_{b}\right\}_{1}^{B}$.
\end{enumerate}

To make a prediction at a new point $x$ :
\begin{itemize}
\item Regression: $\hat{f}_{\mathrm{rf}}^{B}(x)=\frac{1}{B} \sum_{b=1}^{B} T_{b}(x)$.\\
\item Classification: Let $\hat{C}_{b}(x)$ be the class prediction of the $b$ th random tree. Then $\hat{C}_{\mathrm{rf}}^{B}(x)=$ majority vote $\left\{\hat{C}_{b}(x)\right\}_{1}^{B}$.
\end{itemize}

\section{Technical Indicators Used in Analysis}

The following technical indicators were used in. our study. The definitions were scraped from Lo and Hasanhodzic (2010).

\textbf{Standard Deviation} (SD): measures market volatility and is used in statistics to describe the variability or dispersion of a set of data around the average.



\textbf{Relative Strength Index} (RSI): is a momentum indicator used in technical analysis that measures the magnitude of recent price changes to evaluate overbought or oversold conditions in the price of a stock or other asset.


\textbf{Average Directional Index }(ADX): is a technical analysis indicator used by some traders to determine the strength of a trend.


\textbf{Moving Average}(MA): is a stock indicator that is commonly used in technical analysis. The reason for calculating the moving average of a stock is to help smooth out the price data over a specified period of time by creating a constantly updated average price.



\textbf{Exponential Moving Average }(EMA): is a technical chart indicator that tracks the price of an investment (like a stock or commodity) over time.

\textbf{Momentum} (MTM): is a technical indicator which shows the trend direction and measures the pace of the price fluctuation by comparing current and past values.

\textbf{Average True Range} (ATR): is a market volatility indicator used in technical analysis. It is typically derived from the 14-day simple moving average of a series of true range indicators.


\textbf{Bollinger Bands} (BB): are envelopes plotted at a standard deviation level above and below a simple moving average of the price.


\textbf{Stochastic Oscillator} (SO): is a momentum indicator comparing a particular closing price of a security to a range of its prices over a certain period of time.


\textbf{Triple Exponential Average} (TRIX): is an oscillator used to identify oversold and overbought markets, and it can also be used as a momentum indicator.


\textbf{Moving Average Convergence Divergence} (MACD): It is designed to reveal changes in the strength, direction, momentum, and duration of a trend in a stock's price. 



\textbf{Mass Index }(MI): used in technical analysis to predict trend reversals.


\textbf{Vortex Indicator }(VI): is used to spot trend reversals and confirm current trends.


\textbf{True Strength Index} (TSI): is a technical momentum oscillator used to identify trends and reversals.

\textbf{Chaikan Oscillator} (CO): is the difference between the 3-day and 10-day EMAs of the Accumulation Distribution Line.


\textbf{Force Index }(FI): measures the amount of power used to move the price of an asset.

\textbf{Ease of Movement} (EOM): is a technical study that attempts to quantify a mix of momentum and volume information into one value.


\textbf{Commodity Channel Index }(CCI): is a technical indicator that measures the difference between the current price and the historical average price.

\textbf{Keltner Channel }(KC): technical indicator that day traders can use to help assess the current trend and provide trading signals.

\textbf{Ultimate Oscillator} (UO): is a range-bound indicator with a value that fluctuates between 0 and 100.  Used to identify oversold or over bought assets.



% % Appendix A

\chapter{Boosting Grid Search Learning Rates} % Main appendix title







\begin{figure}[h]
    \centering
    \subfloat[Learning Rate: 0.01]{\label{fig:2:a}\includegraphics[width=0.6\textwidth]{/Users/pavansingh/Google Drive (UCT)/STA Honours/Project/Thesis/Python Coding/Figures/RF AND BOOST/VAL/Best_Model_Boosting_0.pdf}}\\
    \subfloat[Learning Rate: 0.1]{\label{fig:2:b}\includegraphics[width=0.6\textwidth]{/Users/pavansingh/Google Drive (UCT)/STA Honours/Project/Thesis/Python Coding/Figures/RF AND BOOST/VAL/Best_Model_Boosting_1.pdf}}    \\
    \subfloat[Learning Rate: 1]{\label{fig:2:d}\includegraphics[width=0.6\textwidth]{/Users/pavansingh/Google Drive (UCT)/STA Honours/Project/Thesis/Python Coding/Figures/RF AND BOOST/VAL/Best_Model_Boosting_2.pdf}} \\
    \caption{\small{Grid search results for using different learning rates on our AdaBoost model.}}
\label{}
\end{figure}


\chapter{Validation and Loss Curves for LSTM} % Main appendix title


\begin{figure}[h]
    \centering
    \subfloat[Multivariate LSTM Ford]{\label{fig:2:a}\includegraphics[width=0.6\textwidth]{/Users/pavansingh/Google Drive (UCT)/STA Honours/Project/Thesis/Python Coding/Figures/ANN/LOSS/ModelLoss_ANN_Ford.pdf}}\\
    \subfloat[Multivariate LSTM Ford]{\label{fig:2:b}\includegraphics[width=0.6\textwidth]{/Users/pavansingh/Google Drive (UCT)/STA Honours/Project/Thesis/Python Coding/Figures/ANN/LOSS/ModelLoss_ANN_Motorola.pdf}}    \\
    \subfloat[Multivariate LSTM Ford]{\label{fig:2:d}\includegraphics[width=0.6\textwidth]{/Users/pavansingh/Google Drive (UCT)/STA Honours/Project/Thesis/Python Coding/Figures/ANN/LOSS/ModelLoss_ANN_Google.pdf}} \\
    \caption{\small{Grid search results for using different learning rates on our AdaBoost model.}}
\label{}
\end{figure}

%----------------------------------------------------------------------------------------
%	BIBLIOGRAPHY
%----------------------------------------------------------------------------------------

%\bibliographystyle{elsarticle-harv}
\bibliographystyle{plain}
%\bibliography{/Users/pavansingh/Google Drive (UCT)/STA Honours/Project/Thesis/Write-Up/Draft/Thesis.bib}
%\nocite{*}
\bibliography{}


%----------------------------------------------------------------------------------------

\end{document}  
