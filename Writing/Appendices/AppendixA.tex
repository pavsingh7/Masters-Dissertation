% Appendix A

\chapter{} % Main appendix title

\section{Random Walk Hypothesis}

The random walk hypothesis states that the single-period log returns, $r_{t}=$ $\log \left(1+R_{t}\right)$, are independent. 

Because
$$
\begin{aligned}
1+R_{t}(k) &=\left(1+R_{t}\right) \cdots\left(1+R_{t-k+1}\right) \\
&=\exp \left(r_{t}\right) \cdots \exp \left(r_{t-k+1}\right) \\
&=\exp \left(r_{t}+\cdots+r_{t-k+1}\right)
\end{aligned}
$$
we have
$$
\log \left\{1+R_{t}(k)\right\}=r_{t}+\cdots+r_{t-k+1}
$$

It is sometimes assumed further that the log returns are $N\left(\mu, \sigma^{2}\right)$ for some constant mean and variance. Since sums of normal random variables are themselves normal, normality of single-period log returns implies normality of multiple-period log returns. Under these assumptions, $\log \left\{1+R_{t}(k)\right\}$ is $N\left(k \mu, k \sigma^{2}\right)$. 



