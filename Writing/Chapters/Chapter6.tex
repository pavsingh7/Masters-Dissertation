% Chapter Template

\chapter{Results} % Main chapter title

\label{Chapter6} % Change X to a consecutive number; for referencing this chapter elsewhere, use \ref{ChapterX}

\section{Results}


% Please add the following required packages to your document preamble:
% \usepackage{multirow}
\begin{table}[h]
 \begin{adjustwidth}{-0.1cm}{}
\begin{tabular}{llllllllll}
\toprule
\multirow{2}{*}{Model} & \multicolumn{3}{c}{Accuracy (\%)} & \multicolumn{3}{c}{F1 Score} & \multicolumn{3}{c}{AUC}         \\
                       & Ford   & Google  & Motorola  & Ford   & Google  & Motorola  & Ford & Google & Motorola \\
                       \midrule
Uni LSTM        & 50.04   & 49.60    & 53.60     & 0.49  & 0.50   & 0.52     & 0.52       &  0.48      &    0.50      \\
Uni FFNN        & 47.20  & 44.80   & 49.60     & 0.48      &  0.46       &   0.50        &  0.47           &  0.49      & 0.47         \\
Random Forest          & 54.40  & 55.20   & 0.52     & 0.52      &   0.54      &   0.48        &   0.60          & 0.59       &  0.56        \\
AdaBoost               & 52.80     & 52.00   & 0.51     & 0.48      & 0.51  &    0.49       &    0.62         & 0.55       &  0.53        \\
Multi LSTM      & 58.40  & 55.20   & 59.20     & 0.60      & 0.53  & 0.63    & 0.65       &    0.58    &  0.56        \\
Multi FFNN      & 55.20  & 53.60   & 52.80     & 0.57      &   0.54      &     0.54      & 0.57            &  0.65      &  0.56        \\
ARIMA                  & 45.60  & 39.20    & 42.40     & 0.44      & 0.38        &  0.38         &             &        &     
 \end{tabular}
 \hline
 \quad
\caption{Results from comparative analysis of various machine learning models on three stocks, using ARIMA as benchmark. The accuracy (in percentage), F1 score and AUC are presented for all the models.}
 \end{adjustwidth}    
 \end{table}

In this paper, we assessed how accurately we can predict the one day ahead stock movement of three companies. The proposed models were trained on 95\% of the entire dataset (approx. 2115 days) and tested on the remaining 125 days. The results of our comparative analysis are shown in Table 6.1. The predicted movements for the Ford Stocks by our final models are summarised in the confusion matrices in figure 6.1. The confusion matrices for Google and Motorola can be found in appendix E. The confusion matrices from our ARIMA model are shown in figure. 6.5. 

Given that stock prediction is a challenging task and a naive random forecast is 50\%, the accuracy achieved by our LSTM models (all above 55\%) is generally reported as a satisfying result for binary stock movement prediction (Nguyen and Shirai, 2015). 

\begin{figure}[h]
    \centering
    \subfloat[Random Forest]{\label{fig:2:a}\includegraphics[width=0.3\textwidth]{/Users/pavansingh/Google Drive (UCT)/STA Honours/Project/Thesis/Python Coding/Figures/RF AND BOOST/CONF/ConfidenceMatrix_RF_Ford.pdf}}
    \subfloat[AdaBoost]{\label{fig:2:b}\includegraphics[width=0.3\textwidth]{/Users/pavansingh/Google Drive (UCT)/STA Honours/Project/Thesis/Python Coding/Figures/RF AND BOOST/CONF/ConfidenceMatrix_Ada_Ford.pdf}}    
    \subfloat[Univariate Feed-Forward]{\label{fig:2:d}\includegraphics[width=0.3\textwidth]{/Users/pavansingh/Google Drive (UCT)/STA Honours/Project/Thesis/Python Coding/Figures/ANN/CONF/ConfidenceMatrix_FFNN_Uni_Ford.pdf}} \\
    \subfloat[Multivariate Feed-Forward]{\label{fig:2:e}\includegraphics[width=0.3\textwidth]{/Users/pavansingh/Google Drive (UCT)/STA Honours/Project/Thesis/Python Coding/Figures/ANN/CONF/ConfidenceMatrix_FFNN_Multi_Ford.pdf}}    
    \subfloat[Univariate LSTM]{\label{fig:2:f}\includegraphics[width=0.3\textwidth]{/Users/pavansingh/Google Drive (UCT)/STA Honours/Project/Thesis/Python Coding/Figures/LSTM/CONF/ConfidenceMatrix_LSTM_Uni_Ford.pdf}}
     \subfloat[Multivariate LSTM]{\label{fig:2:f}\includegraphics[width=0.3\textwidth]{/Users/pavansingh/Google Drive (UCT)/STA Honours/Project/Thesis/Python Coding/Figures/LSTM/CONF/ConfidenceMatrix_LSTM_Multi_Ford.pdf}}
    \caption{\small{Confusion matrixes for the various machine learning models for Ford Motors only. Left axes depicts the true movement of the daily log returns, while the bottom axes depicts the predicted movement}}
\label{}
\end{figure}


Table 6.1 indicates that the LSTM model with all the technical indicators and stock data (multivariate LSTM) performed the best out of all the models, achieving a max accuracy of 59.2\% for predicting the movement of Motorola stock. The model also outperformed every other in accuracy metric for the other two stocks. It is worth noting that the Random Forest model achieved the same accuracy as the multivariate LSTM for the Google stock. The Random Forest, in fact, achieved similar accuracy results to that of the  feed-forward neural network.  The AdaBoost model, performed the worst amongst the multivariate neural networks and ensemble methods. However, it still managed to achieve greater than 50\% accuracy across all stocks. 


The univariate models for the LSTM and FFNN achieved much lower accuracies and F1 scores, than the multivariate models. The univariate FFNN was outperformed by a naive random forecast, indicating the poor predictive performance of the model. This is further supported by the poor AUC scores - below 0.5. Although the univariate LSTM model achieved an accuracy of 54.6\% for the Motorola stock, the model did not perform well against the other stocks - achieving 50.04 and 49.60\%. This indicates that the models were not able to provide anything useful. This also suggests that we cannot achieve any valuable predictive performance in forecasting the one day ahead movement, using previous prices alone as inputs.

 
Given that our models were trying to predict the movement of stocks, a trader would be interested in the F1 score metric, given that it takes into account both false positives and false negatives. The multivariate LSTM model had an F1 score of 0.60 for Ford, 0.53 for Google and 0.63 for Motorola. The F1 score of 0.63 for Motorola and 0.60 for Ford stocks, were in fact the highest values achieved amongst the other models. Notably, the random forest and multivariate FFNN achieved the highest F1 score of 0.54 for Google stock. 

\begin{figure}[!h]
    \centering
    \subfloat[Random Forest]{\label{fig:4:a}\includegraphics[width=0.52\textwidth]{/Users/pavansingh/Google Drive (UCT)/STA Honours/Project/Thesis/Python Coding/Figures/RF AND BOOST/AOC/ROC_AUC_BOTH_RANDOMFOREST.pdf}}
    \subfloat[AdaBoost]{\label{fig:4:b}\includegraphics[width=0.52\textwidth]{/Users/pavansingh/Google Drive (UCT)/STA Honours/Project/Thesis/Python Coding/Figures/RF AND BOOST/AOC/ROC_AUC_BOTH_BOOSTING.pdf}}  
     \caption{\small{ROC Curves for the ensemble decision tree machine learning models for all three stocks. Left axes is the sensitivity, while the bottom axes is the specificity.}}
\label{}
\end{figure}

Based on the computed probabilities from our predictions of the machine learning models, a ROC plot was generated and is shown in Figures 6.2 and 6.3.  ROC-AUC score provides us with information about how well a model is performing its job of separating cases. AUC measures the quality of the model's predictions. Using AUC score as the metric, we now find that the multivariate FFNN and multivariate LSTM had the best classification performance - achieving an AUC score of 0.65 for Google and Ford stocks respectively. As such, our multivariate LSTM model which achieved a AUC score for Ford stock of 0.65 indicates that there is a 65\% that our model can distinguish up movements from down movements of the Ford stock. We also note our Random Forest achieved relatively high AUC scores, indicating the impressive classification performance of the model. 



    
\begin{figure}[h]
    \centering
    \subfloat[Multivariate Feed-Forward]{\label{fig:3:a}\includegraphics[width=0.52\textwidth]{/Users/pavansingh/Google Drive (UCT)/STA Honours/Project/Thesis/Python Coding/Figures/ANN/ROC AUC/ROC_AUC_BOTH_FFNN.pdf}} 
       \subfloat[Multivariate LSTM]{\label{fig:3:b}\includegraphics[width=0.52\textwidth]{/Users/pavansingh/Google Drive (UCT)/STA Honours/Project/Thesis/Python Coding/Figures/LSTM/AUC/ROC_AUC_BOTH_LSTM.pdf}}
    \caption{\small{ROC Curves for neural network machine learning models for all three stocks. Left axes is the sensitivity, while the bottom axes is the specificity.}}
\label{}
\end{figure}


Comparing our machine learning models to that of the benchmark model - ARIMA - table 6.1 indicates that all of our machine learning models achieved accuracy greater than that of the ARIMA model for all three stocks. The benchmark ARIMA model, performed the worst, achieving a max accuracy of 45.60\% for predicting the movement of the log returns for Ford stocks. The ARIMA performed substantially worse that the other multivariate models, for predicting the movement of Google stock - achieving 39.20\%. This result is likely due to under-fitting of the data as the stock returns are complex and are not linear. 



\begin{figure}[!h]
    \centering
    \subfloat[Ford Motors]{\label{fig:2:a}\includegraphics[width=0.3\textwidth]{/Users/pavansingh/Google Drive (UCT)/STA Honours/Project/Thesis/Python Coding/Figures/ARIMA/CONF/ConfidenceMatrix_ARIMA_Ford.pdf}}
    \subfloat[Google]{\label{fig:2:b}\includegraphics[width=0.3\textwidth]{/Users/pavansingh/Google Drive (UCT)/STA Honours/Project/Thesis/Python Coding/Figures/ARIMA/CONF/ConfidenceMatrix_ARIMA_Google.pdf}}   
    \subfloat[Motorola]{\label{fig:2:d}\includegraphics[width=0.3\textwidth]{/Users/pavansingh/Google Drive (UCT)/STA Honours/Project/Thesis/Python Coding/Figures/ARIMA/CONF/ConfidenceMatrix_ARIMA_Motorola.pdf}} 
    \caption{\small{Predicted daily log returns vs the realised log returns for the various machine learning models for Ford Motors only. Left axes depicts the log return while the bottom axes depicts the day}}
\label{}
\end{figure}

\newpage

A noticeable issue shared between all our models, is the poor performance in forecasting the magnitude of the returns. Although our models may have done well to predict the movement of the stocks, we can see that in Figure 6.4 our machine learning models, on average, had very small forecasted returns which were often only a fraction of the size of the realised return. 

\begin{figure}[h]
    \centering
    \subfloat[Multivariate LSTM]{\label{fig:2:a}\includegraphics[width=0.5\textwidth]{/Users/pavansingh/Google Drive (UCT)/STA Honours/Project/Thesis/Python Coding/Figures/LSTM/Returns/Returns_LSTM_Ford.pdf}}
    \subfloat[Multivariate FFNN]{\label{fig:2:b}\includegraphics[width=0.5\textwidth]{/Users/pavansingh/Google Drive (UCT)/STA Honours/Project/Thesis/Python Coding/Figures/ANN/Returns/RETURNS_ANN_FORD.pdf}}    \\
    \subfloat[Random Forest]{\label{fig:2:d}\includegraphics[width=0.5\textwidth]{//Users/pavansingh/Google Drive (UCT)/STA Honours/Project/Thesis/Python Coding/Figures/RF AND BOOST/RETURNS/Returns_RandomForest_Ford.pdf}} 
    \subfloat[AdaBoost]{\label{fig:2:e}\includegraphics[width=0.5\textwidth]{//Users/pavansingh/Google Drive (UCT)/STA Honours/Project/Thesis/Python Coding/Figures/RF AND BOOST/RETURNS/Returns_AdaBoost_Ford.pdf}}    
    \caption{\small{Predicted daily log returns vs the realised log returns for the various machine learning models for Ford Motors only. Left axes depicts the log return while the bottom axes depicts the day}}
\label{}
\end{figure}









