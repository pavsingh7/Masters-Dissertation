% Chapter Template

\chapter{Conclusion} % Main chapter title
\label{Chapter6} % Change X to a consecutive number; for referencing this chapter elsewhere, use \ref{ChapterX}

In Chapter \ref{Chapter5}, we looked into the results from this thesis and their implications concerning our research questions, including a deeper discussion of what was observed from our analysis in general. In this chapter, we aim to offer a comprehensive overview of the primary findings derived from this thesis, representing the culmination of our analysis, and outlining their implications moving forward.

The chapter begins with an overview of the results in Section \ref{sec:6 Summary of Results and Findings}. Following this, Section \ref{sec:6 Limitations} addresses the limitations of the thesis, both from a perspective of limited experimentation, as well as from a perspective of the various assumptions made. These identified shortcomings pave the way for Section \ref{sec:6 Future Work}, which explores potential avenues for further exploration or enhancement of the methodologies employed. Finally, we draw this thesis to a close with Section \ref{sec:6 Conclusion for Conclusion}, which serves to summarise and raise key points to conclude this thesis. 


\section{Summary of Results and Findings}
\label{sec:6 Summary of Results and Findings}


In Section \ref{sec:5 Results}, we evaluated the results from our study for two problems: rating prediction and top-$n$ generation. We summarise the key findings from our study. 



\textbf{Rating Prediction}
\begin{itemize}
    \item The NCF model outperformed all benchmark models across all predictive accuracy metrics, including MAE, RMSE, and MSE.
    \item The review-aware NCF model, integrating ratings and review text, exhibited the best performance, achieving an MAE of 0.490 and RMSE of 0.769.
    \item The inclusion of sentiment analysis alongside reviews marginally decreased performance, resulting in marginally higher RMSE and MAE values (MAE of 0.492 and RMSE of 0.779).
    \item A one-sample paired t-tests showed that there is evidence that the NCF model significantly outperforms all benchmark models at a significance level of $p<0.01$.
    \item Users with more reviews had lower MAE and RMSE values, indicating that the NCF model performed better for users with more reviews.
    \item The predictive accuracy did not change significantly for items with more reviews, indicating that the NCF model performed similarly for items with varying review counts.
    \item  Reviews with fewer words had lower MAE and RMSE values, indicating that the NCF model performed better for reviews with fewer words.
    \end{itemize}
    

\textbf{Top-$N$ Generation}
\begin{itemize}
    \item The NCF model struggled to recommend relevant products for users' top-$n$ recommendations, particularly within the Top-100 list.
    \item All models performed poorly on the top-$n$ tasks, although the NCF model outperformed the benchmarks models.
\end{itemize}


\textbf{Computational Details}
\begin{itemize}
    \item The NCF model with review text and sentiments required only 13 minutes for model training
    \item The slowest NCF model, NCF model with reviews, was quicker (13 minutes) than the fastest benchmark model, user-based collaborative filtering (125 minutes).
    \item NMF took the longest runtime at 457 minutes.
    \item The efficiency of the NCF models was particularly noteworthy with respect to computation, with shorter run-times compared to benchmark models, even when augmented with additional information such as review text and sentiments. Largely attributed to using TensorFlow for NCF, whilst the benchmark models were built from scratch.
\end{itemize}


Directly addressing our research questions, the incorporation of review text enhanced rating prediction performance within recommender systems. However, the addition of sentiments did not yield further improvements. Moreover, the efficacy of the NCF models was evident, performing better than all the benchmark models across the predictive accuracy metrics used. Furthermore, the efficiency of the NCF models was particularly noteworthy with respect to computation, with shorter run-times compared to benchmark models, even when augmented with additional information such as review text and sentiments. 


\section{Limitations}
\label{sec:6 Limitations}

In building our recommender system, our primary focus centered on rating predictions aimed at minimising the disparity between actual and predicted ratings. The overarching goal was to develop a recommendation system capable of accurately predicting unknown ratings for items, leveraging a training set to discern user preferences, and assessing the system's efficacy by evaluating its performance on a testing set containing concealed ratings provided by users for various items. This evaluation process involved measuring predictive accuracy metrics such as RMSE and MAE. Having achieved predicted ratings for all unrated items for each user we were able to identify a list of items (top-$n$) by sorting the ratings for unrated items to identify a top-$n$ list of recommendations for each user. We measured this list using metrics such as recall@$n$ and precision@$n$. The results of which were not impressive. In hindsight, these observations shed light on several limitations inherent in our study, prompting a critical re-evaluation of our methodologies and approaches.

Firstly, we built the model for a specific goal - rating prediction, however we tested its capability on different task - top-$n$ generation. We have established that there is no one universal recommender system \cite{lu2012recommender} and that a recommender's performance is not transferable from one objective to another. Consequently, the results obtained from the top-$n$ generation task offer limited insights into the effectiveness of our review-aware NCF model in this specific use case. Indeed, while our recommender system performed relatively well in predicting ratings for unknown products, its performance in generating top-$n$ recommendations was considerably less satisfactory. That is to say, we observed that the model struggled to recommend relevant (items in test set) products for a users Top-100. In fact, the performance was not better than random recommender. Thus, given the misalignment between the model's training objective and the evaluation task, the results obtained from the top-$n$ generation task should be interpreted with caution, and the conclusions drawn from them should be viewed in light of this limitation. 

To that end, the misalignment between the training objective and the evaluation task underscores the importance of aligning the recommender system's training and evaluation objectives to ensure that the model is optimised for the task at hand. With the evolving understanding that recommendation accuracy alone does not guarantee an effective and satisfying user experience, our approach of generating a recommender solely for predicting unrated items provides a limited scope for building a good or useful recommender system. To address this limitation, it becomes imperative to extend beyond simple accuracy metrics and optimise recommenders for tasks beyond rating prediction. This necessitates a more nuanced evaluation approach that encompasses multiple dimensions of performance, rather than focusing solely on rating prediction accuracy. Such multi-objective recommenders can be designed to optimise for a range of criteria, including diversity, novelty, serendipity, and user satisfaction, thereby ensuring a more comprehensive recommendation capability. 


Another key challenge and limitation encountered in this study pertained to the computational resources allocated to meet the thesis's requirements. For loading and handling the dataset, we relied on our local machine, which presented several challenges in managing the data effectively and efficiently. Thus, we limited our dataset by sampling records from the original repository, which contained over 140 million records. Additionally, we further restricted the dataset to users with 13 or more ratings and items with 13 or more reviews. While these constraints were necessary to ensure the feasibility of our analysis on the local machine, they also introduced limitations that may have impacted the performance and utility of our recommender system. 

With respect to the computational restrictions, beyond dataset handling, the model building and training occurred on our local machine, which inhibited the path of additional experimentation. Despite our efforts, all the benchmark models were restricted (due to their excessive run-times and memory allocations) in their tuning capabilities, with hyperparameter adjustments limited to only two or three options for each parameter. While it is not expected that the performance will be substantially improved had we been able to perform a more extensive hyperparameter search, it could be worthwhile trying to quantify how much improvement can be gained by performing a large hyperparameter search as compared to using a standard set of hyperparameters. Additionally, we compared an NCF model with extensive hyperparameter tuning with benchmark models that had minimal hyperparameter tuning. This comparison could have been more equitable had we performed a more extensive hyperparameter search for the benchmark models. Ultimately, the computational constraints as well as the limited time dictated a lot of the decisions made in the methodology. This challenge also touches upon a broader issue of scalability, an aspect that was acknowledged in Section \ref{subsec:2 Scalability} but was not addressed within the scope of this thesis. Nonetheless, given the context of the recommender system within e-commerce, scalability emerges as a primary concern. Addressing this concern could significantly enhance the run-times of benchmark models, particularly matrix factorisation approaches, by leveraging packages specifically designed to handle the inherent sparsity of the user-item matrix. The benchmark models, detailed in Section \ref{sec:4 Benchmark Models}, were constructed from scratch in Python, with minimal effort directed towards mitigating scalability issues or managing the computational overhead of training. 

One final limitation inherent in our thesis is the omission of addressing the cold start problem. We chose to mitigate the cold start problem, or rather circumvent it entirely, by restricting the dataset to users with 13 or more ratings and items with 13 or more reviews. While this approach provides a starting point, it limits the generalisability of our findings and the applicability of our recommender system to real-world scenarios. The cold start problem is a pervasive challenge in recommender systems, particularly for new users or items with limited interaction history. Addressing this challenge requires the development of innovative strategies to handle users or items with limited historical data. While the cold start problem was not the primary focus of our study, it represents a critical limitation that warrants further exploration and consideration in future research.


\section{Future Work}
\label{sec:6 Future Work}

\subsection{Developing Multi-Objective Recommender Systems}
\label{subsec:6 Multi-Objective Recommender Systems}

We have stablished the importance of a comprehensive evaluation approach to recommender systems, one that extends beyond rating prediction accuracy to encompass a broader range of performance metrics. Avenues for developing multi-objective recommenders that optimise for diverse criteria, including diversity, novelty, serendipity, and user satisfaction, represent a promising direction for future research. By adopting a more holistic evaluation approach, recommender systems can be designed to cater to a wider range of user preferences and needs, thereby enhancing the overall user experience. This approach aligns with the evolving understanding that recommendation accuracy alone does not guarantee an effective and satisfying user experience, underscoring the need for a more nuanced evaluation framework that captures the multifaceted nature of recommender systems. Various strategies can be employed to address this, including direct enhancement of recommendation list diversity and the integration of hybrid recommendation methods to meet different task objectives (\cite{smyth2001similarity};\cite{ziegler2005improving};\cite{hurley2011novelty};\cite{zhou2010solving}). Ultimately, future endeavors should prioritise the adoption of a more comprehensive approach to developing multi-objective recommender systems,and hence, evaluating recommenders based on a broader range of performance metrics.

\subsection{Enhancing Neural Collaborative Filtering with NeuMF}
\label{subsec:6 Enhancing Neural Collaborative Filtering with NeuMF}

The development of our NCF was based off the framework established by \cite{he2017neural}. The findings from our analysis underscored the improvements in predictive accuracy achieved by the NCF model when compared to conventional collaborative filtering methods. The work in this thesis can be extended to hybridising the architecture of NCF by incorporating matrix factorisation, resulting in the algorithm Neural Matrix Factorisation (NeuMF) - as detailed by \cite{he2017neural}. The rationale behind this approach stems from the recognition that traditional matrix factorisation can be viewed as a specialised instance of NCF. Therefore, by fusing the neural interpretation of matrix factorisation with Multilayer Perceptron (MLP), NeuMF emerges as a more generalised model harnessing the linearity of matrix factorisation and the non-linearity of MLP to enhance recommendation quality. Notably, the empirical findings from studies such as \cite{zhang2019deep} and \cite{he2017neural} corroborate the performance benefits offered by NeuMF over NCF. This naturally extends our comparative analysis, inviting us to integrate and adapt the advancements put forth by \cite{he2017neural} to incorporate NeuMF into our framework. This avenue for future can further be extended by investigating the augmentation of review text and sentiments within this enhanced framework — an area that, to the best of our knowledge, has not yet been explored.


\subsection{Exploring Advanced Word Embedding and Sentiment Analysis Techniques}

For our analysis, we employed a simple word embedding technique (USE) to convert review text into numerical vectors, which were subsequently integrated into our NCF model. However, the choice of word embedding technique can significantly impact the performance of the recommender system \cite{asudani2023impact}, with advanced methods such as Word2Vec, GloVe, and BERT offering more sophisticated representations of textual data (\cite{mikolov2013distributed}; \cite{pennington2014glove}; \cite{devlin2018bert}). Therefore, it stands to reason that exploring additional word embedding techniques beyond the simple implementation undertaken in this thesis would represent a natural extension of our research efforts. Similarly, the sentiment analysis component of our study was based on a lexicon-based approaches only, which may not capture the full complexity and nuances of user sentiments. By incorporating more advanced sentiment analysis techniques, such as deep learning-based models, we can possibly enhance the accuracy and granularity of sentiment analysis within our recommender system. Notably, the choice of sentiment analysis technique has been documented as an important determinant in feature creation for subsequent analysis in existing literature \cite{ahuja2019impact}. 

Ultimately, the integration of advanced word embedding and more sophisticated sentiment analysis techniques can enrich the insights derived from user reviews, thereby enhancing the effectiveness of the recommender system. Such an approach can perhaps lead to more representative sentiments from the review text, thereby furnishing the model with richer insights into user preferences. Thus, by experimenting with more methodologies, there is an opportunity to further enhance the effectiveness of our recommender system.



\subsection{Addressing the Cold Start Problem}
\label{subsec:6 Addressing the Cold Start Problem}

A natural extension to our analysis is to address the cold start problem, a challenge in recommender systems that arises when new users or items with limited interaction history are introduced. While we circumvented this issue by restricting our dataset to users with 13 or more ratings and items with 13 or more reviews, this approach limits the generalisability of our findings. To address the cold start problem, innovative strategies can be employed to handle users or items with limited historical data, thereby enhancing the robustness and versatility of the recommender system. For instance, hybrid recommendation approaches that combine collaborative filtering with content-based filtering can be employed to mitigate the cold start problem by leveraging the strengths of both methods. Content-based filtering methods can be leveraged to provide recommendations based on item attributes or user profiles, thereby circumventing the need for historical interaction data \cite{claypool1999combing}. By addressing the cold start problem, the developed recommender systems can cater to a wider range of users and items, thereby enhancing their utility and effectiveness in real-world scenarios. This avenue for future research represents a critical step towards developing more comprehensive and generalisable recommender systems capable of handling the challenges posed by the cold start problem.

\subsection{Leveraging Additional Data Modalities}
\label{subsec:6 Leveraging Additional Data Modalities}

The Amazon Product Review Data (Section \ref{subsec:3 Amazon Review Dataset}) utilised in this study presents additional opportunities for experimentation. Notably, each product within the dataset is accompanied by an image feature — an input that has garnered attention in research within several industries, particularly fashion E-commerce, where images have been leveraged to enrich the top-$n$ generation process (\cite{tuinhof2019image};\cite{kurt2017image}). By augmenting our recommender system to incorporate an additional data modality, such as images, there is opportunity for enhancing the efficacy of the model, building upon the foundation laid in this thesis. Furthermore, the dataset offers a valuable temporal dimension, with each review provided by a user being timestamped, spanning a considerable time frame from 1996 to 2018. Harnessing this temporal aspect and tracking trends in user preferences over time could furnish the recommender system with valuable insights, enabling it to adapt and evolve in response to shifting user preferences. This not only facilitates a more comprehensive evaluation of the recommender system but also ensures its relevance and efficacy under realistic and dynamic conditions. 

\section{Conclusion}
\label{sec:6 Conclusion for Conclusion}


Recommender systems play a crucial role in modern digital landscapes, acting as indispensable tools that guide users towards items they are likely to find valuable based on their individual preferences and the vast array of available options \cite{jannach2010recommender}. By doing so, these systems empower users to efficiently navigate through many options, connecting them with products, services, or knowledge that resonate with their needs and interests \cite{jannach2010recommender}. This ability to streamline the decision-making process has large implications for businesses, ranging from increased cross-selling opportunities to heightened customer satisfaction, ultimately translating into tangible gains in revenue and market competitiveness \cite{leino2007case}.

Collaborative filtering stands as one of the cornerstone paradigms within the realm of recommender systems \cite{burke2015robust}. At its core, collaborative filtering operates on the principle of leveraging user-generated ratings and feedback to generate personalised recommendations. By harnessing the collective 'wisdom' (history) of users who have rated items they've interacted with, this technique effectively identifies patterns and similarities among user preferences \cite{burke2015robust}. This approach, traditionally involved calculating and using similarity metrics, or perhaps matrix factorisation methods to build models to predict interest. However, there has been a growing interest in adapting traditional collaborative filtering framework to incorporate deep learning-based approaches - one such approach is neural collaborative filtering \cite{he2017neural}.


The focal point of this thesis was centered on harnessing a neural network-based approach for collaborative filtering, NCF, with the objective of adapting the framework provided by \cite{he2017neural} to accommodate nuanced user preferences by incorporating review text and sentiments. The primary goals of the thesis were to assess whether the incorporation of a neural network architecture in collaborative filtering enhances performance and whether augmenting a recommender with review text and sentiments yields similar improvements. Leveraging the Amazon product review dataset, extensive text cleaning and pre-processing, including word embedding and sentiment analysis, were conducted to prepare the data for integration into our neural architectures within the recommender models. Adhering to the widely adopted leave-$k$-out approach, we partitioned our data and designated 3 items from each user to form the test set. Our evaluation involved comparing the performance of our NCF systems with that of traditional collaborative filtering methods, namely IBCF, UBCF, as well as NMF. All recommender models were constructed and trained for rating prediction task, with the NCF model featuring a relatively simple MLP neural architecture augmented with additional layers to accommodate additional data inputs. Evaluation metrics such as MAE and RMSE were employed to assess the models' rating prediction capabilities. Additionally, we leveraged predicted ratings to generate top-$n$ recommendations for each user and evaluated the recommender systems' effectiveness using recall@$n$ and precision@$n$ metrics.

The outcomes of this study underscored that incorporating review text into the NCF model enhances its predictive accuracy, with the NCF model with reviews and ratings outperforming all benchmark models across all predictive accuracy metrics. Incorporating sentiments alongside review text did not yield further improvements, with the sentiment-augmented NCF model exhibiting marginally higher RMSE and MAE values. For the top-$n$ generation task, all the models struggled to recommend relevant products for users within the Top-100 list, however, the NCF models outperformed the benchmark models for this task. Additionally, we found that incorporating sentiments alongside review text did not increase the runtimes of the models significantly. 

This thesis concluded by addressing both the limitations inherent in the study and outlined potential avenues for future research that could build upon the current efforts. One of the primary limitations identified was the misalignment between the training objective and the evaluation task, which underscored the importance of aligning the recommender system's training and evaluation objectives to ensure that the model is optimised for the task at hand. To that end, potential avenues for future research include the development of multi-objective recommender systems. In addition, we addressed possibly enhancing the NCF model with NeuMF, exploring advanced word embedding and sentiment analysis techniques, addressing the cold start problem, and leveraging additional data modalities. These avenues represent promising directions for future research that could build upon the current study's findings and enhance the effectiveness and utility of recommender systems in real-world scenarios.

In culmination, this thesis aimed to make incremental contributions to the evolving landscape of recommender systems, with a specific focus on harnessing textual information to augment recommendation accuracy. While our endeavor sought to pave the way for enhanced recommendation accuracy, we acknowledge the hurdles and constraints inherent in crafting recommendation systems capable of adeptly catering to user preferences and item characteristics. The avenues for future work lay bare and offer great potential for expanding the scope of our endeavors and enriching recommendation algorithms.